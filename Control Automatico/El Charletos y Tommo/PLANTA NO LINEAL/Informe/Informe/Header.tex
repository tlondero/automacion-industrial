\documentclass[a4paper]{article}
%\usepackage[utf8]{inputenc}
\usepackage[spanish, es-tabla, es-noshorthands]{babel}
\usepackage[table,xcdraw]{xcolor}
\usepackage[a4paper, footnotesep=1.25cm, headheight=1.25cm, top=2.54cm, left=2.54cm,
 bottom=2.54cm, right=2.54cm]{geometry}
%\geometry{showframe}
 \usepackage[normalem]{ulem}
 \useunder{\uline}{\ul}{}

%VERIFICAR EL HEAD Y EL FOOT EN
%https://ctan.dcc.uchile.cl/macros/latex/contrib/geometry/geometry.pdf

%Paquetes varios:
\usepackage{verbatimbox}

%\usepackage{wrapfig}			%Wrap figure in text
\usepackage[export]{adjustbox}	%Move images
\usepackage{changepage}			%Move tables
\usepackage{todonotes}

\usepackage{tikz}
\usepackage{amsmath}
\usepackage{amsfonts}
\usepackage{amssymb}
\usepackage{float}
\usepackage[graphicx]{realboxes}
\usepackage{caption}
\usepackage{subcaption}
\usepackage{multicol}
\usepackage{multirow}
\setlength{\doublerulesep}{\arrayrulewidth}
%\usepackage{booktabs}

\usepackage{array}
\newcolumntype{C}[1]{>{\centering\let\newline\\\arraybackslash\hspace{0pt}}m{#1}}
%\usepackage[american]{circuitikz}
\usetikzlibrary{calc}
\usepackage{fancyhdr}
\usepackage{units} 

\usepackage{colortbl}
%\usepackage{sectsty}
%\usepackage{unicode-math}

%FONTS (IMPORTANTE): Compilar en XeLaTex o LuaLaTeX
\usepackage{anyfontsize}	%Font size
\usepackage{fontspec}		%Font type
%Si sigue sin andar comentar \usepackage[utf8]{inputenc}
%https://ctan.dcc.uchile.cl/macros/unicodetex/latex/fontspec/fontspec.pdf
%https://www.overleaf.com/learn/latex/XeLaTeX

%Path para imagenes para trabajar en subarchivos
\graphicspath{{../ImagenesPlanta/}{}../ImagenesControl de Estados/}{../ImagenesObservador/}}

%Para el código en el informe
\usepackage{listingsutf8}

%Definiciones de nuevos comandos y colores
%COLORES:
\definecolor{AzulFoot}{rgb}{0.682,0.809,0.926}	%RGB	%{174,206,235}
\definecolor{AzulInfo}{rgb}{0.180,0.455,0.710}	%RGB	%{46,116,181}
\definecolor{AzulTable}{rgb}{0.302,0.507,0.871}	%RGB	%{68,114,196}
\definecolor{PName}{rgb}{0.353,0.353,0.353}		%RGB	%{90,90,90}
\definecolor{mygreen}{rgb}{28,172,0} % color values Red, Green, Blue
\definecolor{mylilas}{rgb}{170,55,241}

%Change Font Size

% #1 = size, #2 = text
\newcommand{\setparagraphsize}[2]{{\fontsize{#1}{6}\selectfont#2 \par}}		%Cambia el size de todo el parrafo
\newcommand{\setlinesize}[2]{{\fontsize{#1}{6}\selectfont#2}}				%Cambia el font de una oración

%IMAGE IN TABLE:			%Ejemplo: \includeintable{.3}{ImagenesFactibilidad/pend}
\renewcommand\fbox{\fcolorbox{white}{white}}
\setlength{\fboxrule}{0pt}	%padding thickness
\setlength{\fboxsep}{4pt}	%border thickness
\newcommand{\includeintable}[2]{	
	\fbox{
		\begin{minipage}{#1\textwidth}
        	\includegraphics[width=\linewidth]{#2}
    	\end{minipage}
	}
}

%LINK IN REF
\newcommand{\reflink}[1]{		%LINK
	\href{#1}{#1}
}

%NOTAS:
\newcommand{\note}[1]{		%RED BIG NOTE (TODO)
	\begin{center}
		\huge{ \textcolor{red}{#1} }
	\end{center}
}

\newcommand{\lnote}[1]{{\fontsize{14}{6}\selectfont\textcolor{green}{#1}}}	%Notas pequeñas

\newcommand{\observacion}[2]{  \ifnumequal{1}{#1}{ { \todo[inline,backgroundcolor=red!25,bordercolor=red!100]{\textbf{Observación: #2}} } }{  }  }

\newcommand{\TBD}{\textcolor{red}{(TBD) }}
\newcommand{\tbd}{\textcolor{red}{(TBD) }}

\newcommand{\TBC}{\textcolor{red}{(TBC) }}
\newcommand{\tbc}{\textcolor{red}{(TBC) }}

\newcommand{\quotes}[1]{``#1''}


\usepackage{hyperref}
\hypersetup{
    colorlinks=true,
    linkcolor=black,
    filecolor=magenta,      
    urlcolor=AzulInfo,
    citecolor=AzulInfo,    
}

%Configuración del header y del footer:
\usepackage{etoolbox}
\pagestyle{fancy}
\fancyhf{}
\rfoot{\thepage}
\renewcommand{\footrulewidth}{4pt}
\renewcommand{\headrulewidth}{0pt}
\patchcmd{\footrule}{\hrule}{\color{AzulFoot}\hrule}{}{}

\usepackage{titlesec}		%Para hacer las subsubsubsections

%Colores a los nombres de las secciones:
%\sectionfont{\color{AzulInfo}}  % sets color of sections
%\subsectionfont{\color{AzulInfo}}
%\subsubsectionfont{\color{AzulInfo}}

%PICTURES AND TABLE INDEX:
\newcommand{\Section}[1]{ \section{#1} 
	\phantomsection \setcounter{figure}{0} \setcounter{table}{0} \renewcommand{\thetable}{\arabic{section}.\arabic{table}} \renewcommand{\thefigure}{\arabic{section}.\arabic{figure}}
}

\newcommand{\Subsection}[1]{ \subsection{#1}
	\phantomsection \setcounter{figure}{0} \setcounter{table}{0} \renewcommand{\thetable}{\arabic{section}.\arabic{subsection}.\arabic{table}} \renewcommand{\thefigure}{\arabic{section}.\arabic{subsection}.\arabic{figure}}
}

\newcommand{\Subsubsection}[1]{ \subsubsection{#1} 
	\phantomsection \setcounter{figure}{0} \setcounter{table}{0} \renewcommand{\thetable}{\arabic{section}.\arabic{subsection}.\arabic{subsubsection}.\arabic{table}} \renewcommand{\thefigure}{\arabic{section}.\arabic{subsection}.\arabic{subsubsection}.\arabic{figure}}
}

%Definición de subsubsubsection:
\titleclass{\subsubsubsection}{straight}[\subsection]

\newcounter{subsubsubsection}[subsubsection]
\renewcommand\thesubsubsubsection{\thesubsubsection.\arabic{subsubsubsection}}

\titleformat{\subsubsubsection}
  {\normalfont\normalsize\bfseries\color{AzulInfo}}{\thesubsubsubsection}{1em}{}	%Color de subsubsubsection
\titlespacing*{\subsubsubsection}
{0pt}{3.25ex plus 1ex minus .2ex}{1.5ex plus .2ex}

\makeatletter
\renewcommand\paragraph{\@startsection{paragraph}{5}{\z@}%
  {3.25ex \@plus1ex \@minus.2ex}%
  {-1em}%
  {\normalfont\normalsize\bfseries}}
\renewcommand\subparagraph{\@startsection{subparagraph}{6}{\parindent}%
  {3.25ex \@plus1ex \@minus .2ex}%
  {-1em}%
  {\normalfont\normalsize\bfseries}}
\def\toclevel@subsubsubsection{4}
\def\toclevel@paragraph{5}
\def\toclevel@paragraph{6}
\def\l@subsubsubsection{\@dottedtocline{4}{7em}{4em}}
\def\l@paragraph{\@dottedtocline{5}{10em}{5em}}
\def\l@subparagraph{\@dottedtocline{6}{14em}{6em}}
\makeatother

\setcounter{secnumdepth}{4}
\setcounter{tocdepth}{4}

%Subsubsubsection:
\newcommand{\Subsubsubsection}[1]{ \subsubsubsection{#1} 
	\phantomsection \setcounter{figure}{0} \setcounter{table}{0} \renewcommand{\thetable}{\arabic{section}.\arabic{subsection}.\arabic{subsubsection}.\arabic{subsubsubsection}.\arabic{table}} \renewcommand{\thefigure}{\arabic{section}.\arabic{subsection}.\arabic{subsubsection}.\arabic{subsubsubsection}.\arabic{figure}}
}

%Tamaño, color e identación de sección, subsección, subsubsección y subsubsubsección:
%La identación de las subsecciones está tambien en Index-cfg.tex para el toc, lot y lot en el index
\titleformat{\section}[block]{\fontsize{16}{6}\selectfont\bfseries\color{AzulInfo}}{\thesection.}{1em}{} 
\titleformat{\subsection}[block]{\hspace{2.5em}\fontsize{13}{6}\selectfont\color{AzulInfo}}{\thesubsection}{1em}{}
\titleformat{\subsubsection}[block]{\hspace{3.5em}\fontsize{12}{6}\selectfont\color{AzulInfo}}{\thesubsubsection}{1em}{}
\titleformat{\subsubsubsection}[block]{\hspace{4em}\fontsize{11}{6}\selectfont\color{AzulInfo}}{\thesubsubsubsection}{1em}{}

%Pone las refrencias en el indice
\usepackage[numbib, nottoc, notlot, notlof]{tocbibind}

%Pone toc, lof y lot en colores y elijo el titulo de estos
\addto\captionsspanish{
	\renewcommand\contentsname{\hspace{2.5em}\fontsize{13}{6} \arabic{section}.\arabic{subsection}.\hspace{5mm}Contenidos}
	\renewcommand\listfigurename{\hspace{2.5em}\fontsize{13}{6} \arabic{section}.\arabic{subsection}.\hspace{5mm}Lista de Figuras}
	\renewcommand\listtablename{\hspace{2.5em}\fontsize{13}{6} \arabic{section}.\arabic{subsection}.\hspace{5mm}Lista de Tablas}
}

\newcommand{\MyIndex}{
	\Section{Indice}

	%Agrega TOC al indice
	\stepcounter{subsection}	
	\addcontentsline{toc}{subsection}{\protect\numberline{\thesubsection}{Contenidos}}
	\tableofcontents

	%Agrega LOF al indice
	\stepcounter{subsection}
	\addcontentsline{toc}{subsection}{\protect\numberline{\thesubsection}{Lista de Figuras}}
	\listoffigures

	%Agrega LOT al indice
%	\stepcounter{subsection}
%	\addcontentsline{toc}{subsection}{\protect\numberline{\thesubsection}{Lista de Tablas}}
%	\listoftables
}

%Indices: cambio la separación de los numeros para que entren tablas y fotos
\usepackage{tocloft}
\setlength{\cftfignumwidth}{1.35cm}  % change numwidth from figures in lof
\setlength{\cfttabnumwidth}{1.35cm}  % change numwidth from tables in lot
\renewcommand{\cfttoctitlefont}{\Large\bfseries\color{AzulInfo}}
\renewcommand{\cftloftitlefont}{\Large\bfseries\color{AzulInfo}}
\renewcommand{\cftlottitlefont}{\Large\bfseries\color{AzulInfo}}

%Coloca lineas punteadas a las seciones en el TOC
\renewcommand{\cftsecleader}{\cftdotfill{\cftdotsep}}

%Items con bullets y no cuadrados
\renewcommand{\labelitemi}{\textbullet }

%Código eninforme
\lstset{language=Matlab,%
    %basicstyle=\color{red},
    breaklines=true,%
    morekeywords={matlab2tikz},
    keywordstyle=\color{blue},%
    morekeywords=[2]{1}, keywordstyle=[2]{\color{black}},
    identifierstyle=\color{black},%
    stringstyle=\color{mylilas},
    commentstyle=\color{mygreen},%
    showstringspaces=false,%without this there will be a symbol in the places where there is a space
    numbers=left,%
    numberstyle={\tiny \color{black}},% size of the numbers
    numbersep=9pt, % this defines how far the numbers are from the text
    emph=[1]{for,end,break},emphstyle=[1]\color{red}, %some words to emphasise
    %emph=[2]{word1,word2}, emphstyle=[2]{style},    
    literate=
    	{á}{{\'a}}1 {é}{{\'e}}1 {í}{{\'i}}1 {ó}{{\'o}}1 {ú}{{\'u}}1
  		{Á}{{\'A}}1 {É}{{\'E}}1 {Í}{{\'I}}1 {Ó}{{\'O}}1 {Ú}{{\'U}}1
		{à}{{\`a}}1 {è}{{\`e}}1 {ì}{{\`i}}1 {ò}{{\`o}}1 {ù}{{\`u}}1
  		{À}{{\`A}}1 {È}{{\'E}}1 {Ì}{{\`I}}1 {Ò}{{\`O}}1 {Ù}{{\`U}}1
  		{ä}{{\"a}}1 {ë}{{\"e}}1 {ï}{{\"i}}1 {ö}{{\"o}}1 {ü}{{\"u}}1
  		{Ä}{{\"A}}1 {Ë}{{\"E}}1 {Ï}{{\"I}}1 {Ö}{{\"O}}1 {Ü}{{\"U}}1
  		{â}{{\^a}}1 {ê}{{\^e}}1 {î}{{\^i}}1 {ô}{{\^o}}1 {û}{{\^u}}1
  		{Â}{{\^A}}1 {Ê}{{\^E}}1 {Î}{{\^I}}1 {Ô}{{\^O}}1 {Û}{{\^U}}1
  		{Ã}{{\~A}}1 {ã}{{\~a}}1 {Õ}{{\~O}}1 {õ}{{\~o}}1
  		{œ}{{\oe}}1 {Œ}{{\OE}}1 {æ}{{\ae}}1 {Æ}{{\AE}}1 {ß}{{\ss}}1
  		{ű}{{\H{u}}}1 {Ű}{{\H{U}}}1 {ő}{{\H{o}}}1 {Ő}{{\H{O}}}1
  		{ç}{{\c c}}1 {Ç}{{\c C}}1 {ø}{{\o}}1 {å}{{\r a}}1 {Å}{{\r A}}1
  		{€}{{\euro}}1 {£}{{\pounds}}1 {«}{{\guillemotleft}}1
  		{»}{{\guillemotright}}1 {ñ}{{\~n}}1 {Ñ}{{\~N}}1 {¿}{{?`}}1,
  	inputencoding=latin1,
}
