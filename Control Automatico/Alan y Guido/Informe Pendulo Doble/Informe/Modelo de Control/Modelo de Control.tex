%\documentclass[a4paper]{article}
\usepackage[utf8]{inputenc}
\usepackage[spanish, es-tabla, es-noshorthands]{babel}
\usepackage[table,xcdraw]{xcolor}
\usepackage[a4paper, footnotesep=1.25cm, headheight=1.25cm, top=2.54cm, left=2.54cm, bottom=2.54cm, right=2.54cm]{geometry}
%\geometry{showframe}

%\usepackage{wrapfig}			%Wrap figure in text
\usepackage[export]{adjustbox}	%Move images
\usepackage{changepage}			%Move tables

\usepackage{tikz}
\usepackage{amsmath}
\usepackage{amsfonts}
\usepackage{amssymb}
\usepackage{float}
\usepackage{graphicx}
\usepackage{caption}
\usepackage{subcaption}
\usepackage{multicol}
\usepackage{multirow}
\usepackage{wrapfig}
\setlength{\doublerulesep}{\arrayrulewidth}
\usepackage{booktabs}
\usepackage[numbib, nottoc, notlot, notlof]{tocbibind}

\usepackage{hyperref}
\hypersetup{
    colorlinks=true,
    linkcolor=blue,
    filecolor=magenta,      
    urlcolor=blue,
    citecolor=blue,    
}

%Change Font Size

% #1 = size, #2 = text
\newcommand{\setparagraphsize}[2]{{\fontsize{#1}{6}\selectfont#2 \par}}		%Cambia el size de todo el parrafo
\newcommand{\setlinesize}[2]{{\fontsize{#1}{6}\selectfont#2}}				%Cambia el font de una oración

\newcommand{\note}[1]{
	\begin{center}
		\huge{ \textcolor{red}{#1} }
	\end{center}
}

%FONTS (IMPORTANTE): Compilar en XeLaTex o LuaLaTeX
\usepackage{anyfontsize}	%Font size
\usepackage{fontspec}		%Font type

\usepackage{etoolbox}
\usepackage{todonotes}

\newcommand{\observacion}[2]{  \ifnumequal{1}{#1}{ { \todo[inline,backgroundcolor=red!25,bordercolor=red!100]{\textbf{Observación: #2}} } }{  }  }

\setcounter{topnumber}{2}
\setcounter{bottomnumber}{2}
\setcounter{totalnumber}{4}
\renewcommand{\topfraction}{0.85}
\renewcommand{\bottomfraction}{0.85}
\renewcommand{\textfraction}{0.15}
\renewcommand{\floatpagefraction}{0.8}
\renewcommand{\textfraction}{0.1}
\setlength{\floatsep}{5pt plus 2pt minus 2pt}
\setlength{\textfloatsep}{5pt plus 2pt minus 2pt}
\setlength{\intextsep}{5pt plus 2pt minus 2pt}

\newcommand{\quotes}[1]{``#1''}
\usepackage{array}
\newcolumntype{C}[1]{>{\centering\let\newline\\\arraybackslash\hspace{0pt}}m{#1}}
\usepackage[american]{circuitikz}
\usetikzlibrary{calc}
\usepackage{fancyhdr}
\usepackage{units} 

\graphicspath{{../Control de posición no lineal/}{../Control de fuerza no lineal/}{../Control híbrido no lineal/}{../Referencias/}{../Deducción de modelo/}{../Conclusiones/}}

\pagestyle{fancy}
\fancyhf{}
\lhead{22.99 - Automación Industrial}
\rhead{Lambertucci, Londero B., Maselli, Mechoulam}
\rfoot{Página \thepage}

%Items con bullets y no cuadrados
\renewcommand{\labelitemi}{\textbullet }

%%
%\begin{document}

\Subsection{Espacio de estados}
Luego de hacer la linealización del sistema :
\begin{equation}
 A = \begin{bmatrix}
0 &  0 & 0 & 1 &  0 & 0\\

0 &  0 & 0 & 0 &  1 & 0\\
0 &  0 & 0 & 0 &  0 & 1\\

0 &  -\frac{3g(2m_1^2 + 5m_1m_2 + 2m_2^2)}{2(4m_0m_1 + 3m_0m_2 + m_1m_2 + m_1^2} & 
\frac{3gm_1m_2}{2(4m_0m_1 + 3m_0m_2 + m_1m_2 + m_1^2}  & 0 &  0 & 0\\

0 &  \frac{3g(4m_1^2 + 9m_1m_2 + 4m_0m_1 + 2m_2^2 + 8m_0m_2)}{2L_1(4m_0m_1 + 3m_0m_2 + m_1m_2 + m_1^2)} & -\frac{9*g*(2m_0m_2 + m_1m_2)}{2L_1(4m_0m_1 + 3m_0m_2 + m_1m_2 + m_1^2)}
 & 0 &  0 & 0\\

0 &   -\frac{9g(2m_0m_1 + 4m_0m_2 + 2m_1m_2 + m_1^2)}{2L_2(4m_0m_1 + 3m_0m_2 + m_1m_2 + m_1^2)} & \frac{3g(4m_0m_1 + 12m_0m_2 + 4m_1m_2 + m_1^2)}{2L_2(4m_0m_1 + 3m_0m_2 + m_1m_2 + m_1^2)} & 0 &  0 & 0
\end{bmatrix}
\end{equation}
\begin{equation}
 B = \begin{bmatrix}
0 \\
0 \\
0 \\
\frac{4m_1 + 3m_2}{4m_0m_1 + 3m_0m_2 + m_1m_2 + m_1^2} \\
 -\frac{3(2m_1 + m_2)}{L_1(4m_0m_1 + 3m_0m_2 + m_1m_2 + m_1^2)} \\
\frac{2m_2}{L_2(4m_0m_1 + 3m_0m_2 + m_1m_2 + m_1^2)}
\end{bmatrix}
\end{equation}
Si se opta por los siguientes valores:
\begin{equation}
m_0 = 5 \ Kg
\hspace{0.5cm}
m_1 = 1 \ Kg
\hspace{0.5cm}
m_2 = 1 \ Kg
\hspace{0.5cm}
L_1 = 1 \ m
\hspace{0.5cm}
L_2 = 1.5 \ m
\hspace{0.5cm}
g= 9.8 \ \frac{m}{s^2}
\end{equation}
\
Se obtiene:

\begin{equation}
 A = \begin{bmatrix}
0 &  0 & 0 & 1 &  0 & 0\\
0 &  0 & 0 & 0 &  1 & 0\\
0 &  0 & 0 & 0 &  0 & 1\\
0 &  -3.5757 & 0.3973 & 0 &  0 & 0\\
0 &  29.7973 & -13.1108 & 0 &  0 & 0\\
0 &  -26.2216 & 22.5135 & 0 &  0 & 0
\end{bmatrix}
\end{equation}
\begin{equation}
 B = \begin{bmatrix}
0 \\
0 \\
0 \\
0.1892 \\
-0.2432 \\
0.036 
\end{bmatrix}
\end{equation}

\Subsection{Controlabilidad y Observabilidad}

Se define al set de estados alcanzables $\mathcal{R}_t$ en un tiempo $t$ como aquellos estados en los que el sistema se puede encontrar. Por otro lado, se define al subespacio controlable $\mathcal{C}_{AB}$ como aquellos estados a los que se puede forzar el sistema mediante una entrada $u(t)$ apropiada. Se puede probar que para $t > 0$ el set de estados alcanzables $\mathcal{R}_t$ es igual que el subespacio controlable $\mathcal{C}_{AB}$.

Se dice que el par $(A, B)$ es controlable, y por ende un sistema definido con esas matrices es controlable si la matriz de controlabilidad

\begin{equation}
[B \ AB \ \cdots \ A^{n-1}B]
\end{equation}

es de rango completo.

Para el caso del péndulo doble, se tiene que

\begin{equation}
[B \ AB \ \cdots \ A^{n-1}B] = 	\begin{bmatrix}
0       &  0.1892  & 0        & 0.8839  &  0         & 30.4554\\
0       &  -0.2432 & 0        & -7.7189 &  0         & -324.2311\\
0       &  0.0360  & 0        & 7.1876  &  0         & 364.2140\\
0.1892  &  0       & 0.8839   & 0       &  30.4554   & 0\\
-0.2432 &  0       & -7.7189  & 0       &  -324.2311 & 0\\
0.0360  &  0       & 7.1876   & 0       &  364.2140  & 0
\end{bmatrix}
\end{equation}

donde se puede observar que la matriz es de rango completo, por lo que el sistema es controlable.

%Otro método más sofisticado para probar la controlabilidad de un sistema es el Test PBH, el cual se basa 

Otros dos aspectos importantes del sistema son la detectabilidad y la observabilidad, dado que en la vida real muchas veces no es posible medir todas las variables del sistema. 

El estudio de la observabilidad del sistema se basa en comprobar la posibilidad de estimar las variables de estado a partir de la salida. Por otro lado, un estudio más débil pero de igual importancia teórica es la detectabilidad. Un sistema es detectable si todos sus estados no observables son estables. Se puede probar que si el par $(C*,A*)$ es controlable, entonces el par $(A,C)$ es observable. 

En el caso del problema estudiado, se tiene que, tomando

\begin{equation}
C = 	\begin{bmatrix}
1 & 0 & 0 & 0 & 0 & 0\\
0 & 1 & 0 & 0 & 0 & 0\\
0 & 0 & 1 & 0 & 0 & 0
\end{bmatrix}
\end{equation}

luego

\begin{equation}
rank[C* \ A*C* \ \cdots \ A*^{n-1}C*]
\end{equation}

es de rango completo, por lo que el sistema será observable midiendo las variables de estado definidas por la matriz C.


\Subsection{Realimentaci\'on de Estados}


\Subsection{Observador}
\Subsection{Discretizaci\'on}
%\end{document}