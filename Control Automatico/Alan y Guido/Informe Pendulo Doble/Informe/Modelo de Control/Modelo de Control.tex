%\documentclass[a4paper]{article}
\usepackage[utf8]{inputenc}
\usepackage[spanish, es-tabla, es-noshorthands]{babel}
\usepackage[table,xcdraw]{xcolor}
\usepackage[a4paper, footnotesep=1.25cm, headheight=1.25cm, top=2.54cm, left=2.54cm, bottom=2.54cm, right=2.54cm]{geometry}
%\geometry{showframe}

%\usepackage{wrapfig}			%Wrap figure in text
\usepackage[export]{adjustbox}	%Move images
\usepackage{changepage}			%Move tables

\usepackage{tikz}
\usepackage{amsmath}
\usepackage{amsfonts}
\usepackage{amssymb}
\usepackage{float}
\usepackage{graphicx}
\usepackage{caption}
\usepackage{subcaption}
\usepackage{multicol}
\usepackage{multirow}
\usepackage{wrapfig}
\setlength{\doublerulesep}{\arrayrulewidth}
\usepackage{booktabs}
\usepackage[numbib, nottoc, notlot, notlof]{tocbibind}

\usepackage{hyperref}
\hypersetup{
    colorlinks=true,
    linkcolor=blue,
    filecolor=magenta,      
    urlcolor=blue,
    citecolor=blue,    
}

%Change Font Size

% #1 = size, #2 = text
\newcommand{\setparagraphsize}[2]{{\fontsize{#1}{6}\selectfont#2 \par}}		%Cambia el size de todo el parrafo
\newcommand{\setlinesize}[2]{{\fontsize{#1}{6}\selectfont#2}}				%Cambia el font de una oración

\newcommand{\note}[1]{
	\begin{center}
		\huge{ \textcolor{red}{#1} }
	\end{center}
}

%FONTS (IMPORTANTE): Compilar en XeLaTex o LuaLaTeX
\usepackage{anyfontsize}	%Font size
\usepackage{fontspec}		%Font type

\usepackage{etoolbox}
\usepackage{todonotes}

\newcommand{\observacion}[2]{  \ifnumequal{1}{#1}{ { \todo[inline,backgroundcolor=red!25,bordercolor=red!100]{\textbf{Observación: #2}} } }{  }  }

\setcounter{topnumber}{2}
\setcounter{bottomnumber}{2}
\setcounter{totalnumber}{4}
\renewcommand{\topfraction}{0.85}
\renewcommand{\bottomfraction}{0.85}
\renewcommand{\textfraction}{0.15}
\renewcommand{\floatpagefraction}{0.8}
\renewcommand{\textfraction}{0.1}
\setlength{\floatsep}{5pt plus 2pt minus 2pt}
\setlength{\textfloatsep}{5pt plus 2pt minus 2pt}
\setlength{\intextsep}{5pt plus 2pt minus 2pt}

\newcommand{\quotes}[1]{``#1''}
\usepackage{array}
\newcolumntype{C}[1]{>{\centering\let\newline\\\arraybackslash\hspace{0pt}}m{#1}}
\usepackage[american]{circuitikz}
\usetikzlibrary{calc}
\usepackage{fancyhdr}
\usepackage{units} 

\graphicspath{{../Control de posición no lineal/}{../Control de fuerza no lineal/}{../Control híbrido no lineal/}{../Referencias/}{../Deducción de modelo/}{../Conclusiones/}}

\pagestyle{fancy}
\fancyhf{}
\lhead{22.99 - Automación Industrial}
\rhead{Lambertucci, Londero B., Maselli, Mechoulam}
\rfoot{Página \thepage}

%Items con bullets y no cuadrados
\renewcommand{\labelitemi}{\textbullet }

%%
%\begin{document}

\Subsection{Espacio de estados}
Luego de hacer la linealización del sistema :
\begin{equation}
 A = \begin{bmatrix}
0 &  0 & 0 & 1 &  0 & 0\\

0 &  0 & 0 & 0 &  1 & 0\\
0 &  0 & 0 & 0 &  0 & 1\\

0 &  -\frac{3g(2m_1^2 + 5m_1m_2 + 2m_2^2)}{2(4m_0m_1 + 3m_0m_2 + m_1m_2 + m_1^2} & 
\frac{3gm_1m_2}{2(4m_0m_1 + 3m_0m_2 + m_1m_2 + m_1^2}  & 0 &  0 & 0\\

0 &  \frac{3g(4m_1^2 + 9m_1m_2 + 4m_0m_1 + 2m_2^2 + 8m_0m_2)}{2L_1(4m_0m_1 + 3m_0m_2 + m_1m_2 + m_1^2)} & -\frac{9*g*(2m_0m_2 + m_1m_2)}{2L_1(4m_0m_1 + 3m_0m_2 + m_1m_2 + m_1^2)}
 & 0 &  0 & 0\\

0 &   -\frac{9g(2m_0m_1 + 4m_0m_2 + 2m_1m_2 + m_1^2)}{2L_2(4m_0m_1 + 3m_0m_2 + m_1m_2 + m_1^2)} & \frac{3g(4m_0m_1 + 12m_0m_2 + 4m_1m_2 + m_1^2)}{2L_2(4m_0m_1 + 3m_0m_2 + m_1m_2 + m_1^2)} & 0 &  0 & 0
\end{bmatrix}
\end{equation}
\begin{equation}
 B = \begin{bmatrix}
0 \\
0 \\
0 \\
\frac{4m_1 + 3m_2}{4m_0m_1 + 3m_0m_2 + m_1m_2 + m_1^2} \\
 -\frac{3(2m_1 + m_2)}{L_1(4m_0m_1 + 3m_0m_2 + m_1m_2 + m_1^2)} \\
\frac{2m_2}{L_2(4m_0m_1 + 3m_0m_2 + m_1m_2 + m_1^2)}
\end{bmatrix}
\end{equation}
Si se opta por los siguientes valores:
\begin{equation}
m_0 = 5 \ Kg
\hspace{0.5cm}
m_1 = 1 \ Kg
\hspace{0.5cm}
m_2 = 1 \ Kg
\hspace{0.5cm}
L_1 = 1 \ m
\hspace{0.5cm}
L_2 = 1.5 \ m
\hspace{0.5cm}
g= 9.8 \ \frac{m}{s^2}
\end{equation}
\
Se obtiene:

\begin{equation}
 A = \begin{bmatrix}
0 &  0 & 0 & 1 &  0 & 0\\
0 &  0 & 0 & 0 &  1 & 0\\
0 &  0 & 0 & 0 &  0 & 1\\
0 &  -3.5757 & 0.3973 & 0 &  0 & 0\\
0 &  29.7973 & -13.1108 & 0 &  0 & 0\\
0 &  -26.2216 & 22.5135 & 0 &  0 & 0
\end{bmatrix}
\end{equation}
\begin{equation}
 B = \begin{bmatrix}
0 \\
0 \\
0 \\
0.1892 \\
-0.2432 \\
0.036 
\end{bmatrix}
\end{equation}
\Subsection{Controlabilidad y Observabilidad}
Hablar de alcanzabilidad detectabilidad
Observabilidad controlabilidad 
Test PBH
\Subsection{Realimentaci\'on de Estados}
\Subsection{Observador}
\Subsection{Discretizaci\'on}
%\end{document}