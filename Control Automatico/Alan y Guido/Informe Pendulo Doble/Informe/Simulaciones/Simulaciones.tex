%\documentclass[a4paper]{article}
\usepackage[utf8]{inputenc}
\usepackage[spanish, es-tabla, es-noshorthands]{babel}
\usepackage[table,xcdraw]{xcolor}
\usepackage[a4paper, footnotesep = 1cm, width=22cm, top=2.5cm, height=25cm, textwidth=20cm, textheight=25cm]{geometry}
%\geometry{showframe}

\usepackage{tikz}
\usepackage{amsmath}
\usepackage{amsfonts}
\usepackage{amssymb}
\usepackage{float}
\usepackage{graphicx}
\usepackage{caption}
\usepackage{subcaption}
\usepackage{multicol}
\usepackage{multirow}
\usepackage{wrapfig}
\setlength{\doublerulesep}{\arrayrulewidth}
\usepackage{booktabs}

\usepackage{hyperref}
\hypersetup{
    colorlinks=true,
    linkcolor=blue,
    filecolor=magenta,      
    urlcolor=blue,
    citecolor=blue,    
}

\newcommand{\note}[1]{
	\begin{center}
		\huge{ \textcolor{red}{#1} }
	\end{center}
}

\setcounter{topnumber}{2}
\setcounter{bottomnumber}{2}
\setcounter{totalnumber}{4}
\renewcommand{\topfraction}{0.85}
\renewcommand{\bottomfraction}{0.85}
\renewcommand{\textfraction}{0.15}
\renewcommand{\floatpagefraction}{0.8}
\renewcommand{\textfraction}{0.1}
\setlength{\floatsep}{5pt plus 2pt minus 2pt}
\setlength{\textfloatsep}{5pt plus 2pt minus 2pt}
\setlength{\intextsep}{5pt plus 2pt minus 2pt}

\newcommand{\quotes}[1]{``#1''}
\usepackage{array}
\newcolumntype{C}[1]{>{\centering\let\newline\\\arraybackslash\hspace{0pt}}m{#1}}
\usepackage[american]{circuitikz}
\usetikzlibrary{calc}
\usepackage{fancyhdr}
\usepackage{units} 

\graphicspath{{../Ejercicio-1/}{../Ejercicio-2/}{../Ejercicio-3/}{../Ejercicio-4/}}

\pagestyle{fancy}
\fancyhf{}
\lhead{22.14 - Electrónica IV}
\rhead{Mechoulam, Lambertucci, Londero}
\rfoot{Página \thepage}

%
%\begin{document}

\Subsection{Modelo de Simscape}

Para lograr mayor apego a la realidad, se decidió utilizar un modelo con fricción obtenido utilizando el framework de Simscape de Simulink, que se pueden observar en las Figuras (\ref{fig:simscape}) y (\ref{fig:simscape_dp}).

Los valores de fricción que se utilizaron para cada joint son de $0.004 \ \frac{N\cdot s}{m}$, resultando así en las siguientes matrices:

\begin{equation}
 A_{simscape} = \begin{bmatrix}
0 &  0 & 0 & 1 &  0 & 0\\
0 &  0 & 0 & 0 &  1 & 0\\
0 &  0 & 0 & 0 &  0 & 1\\
0 &  -3.1784 & 0.3973 & -0.0008 &  0.0557 & -0.0681\\
0 &  16.6865 & -13.1108 & 0.001 &  -0.4646 & 0.8734\\
0 &  -20.3946 & 35.6243 & -0.0012 &  0.8734 & -1.9842
\end{bmatrix}
\end{equation}
\begin{equation}
 B_{simscape} = \begin{bmatrix}
0 \\
0 \\
0 \\
0.1892 \\
-0.2432 \\
0.2973 
\end{bmatrix}
\end{equation}

de ahora en más llamadas simplemente A y B. Estas matrices son muy similares a las obtenidas teóricamente, con la excepción más notoria del corrimiento de uno de los dos polos en el origen hacia el semiplano izquierdo.

\begin{figure}[H]
	\centering
	\includegraphics[width=1.1\linewidth]{../Simulaciones/ImagenesSimulaciones/simscape.png}
	\caption{Bloques de la simulación del péndulo doble realizada con el framework de Simscape en Simulink.}	
	\label{fig:simscape}
\end{figure}
\begin{figure}[H]
	\centering
	\includegraphics[width=0.7\linewidth]{../Simulaciones/ImagenesSimulaciones/simscape_double_pendulum.png}
	\caption{Simulación del péndulo doble realizada con el framework de Simscape en Simulink.}	
	\label{fig:simscape_dp}
\end{figure}

\Subsection{Simulink}
%\end{document}