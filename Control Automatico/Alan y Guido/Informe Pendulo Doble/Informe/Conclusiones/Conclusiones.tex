%\documentclass[a4paper]{article}
\usepackage[utf8]{inputenc}
\usepackage[spanish, es-tabla, es-noshorthands]{babel}
\usepackage[table,xcdraw]{xcolor}
\usepackage[a4paper, footnotesep=1.25cm, headheight=1.25cm, top=2.54cm, left=2.54cm, bottom=2.54cm, right=2.54cm]{geometry}
%\geometry{showframe}

%\usepackage{wrapfig}			%Wrap figure in text
\usepackage[export]{adjustbox}	%Move images
\usepackage{changepage}			%Move tables

\usepackage{tikz}
\usepackage{amsmath}
\usepackage{amsfonts}
\usepackage{amssymb}
\usepackage{float}
\usepackage{graphicx}
\usepackage{caption}
\usepackage{subcaption}
\usepackage{multicol}
\usepackage{multirow}
\usepackage{wrapfig}
\setlength{\doublerulesep}{\arrayrulewidth}
\usepackage{booktabs}
\usepackage[numbib, nottoc, notlot, notlof]{tocbibind}

\usepackage{hyperref}
\hypersetup{
    colorlinks=true,
    linkcolor=blue,
    filecolor=magenta,      
    urlcolor=blue,
    citecolor=blue,    
}

%Change Font Size

% #1 = size, #2 = text
\newcommand{\setparagraphsize}[2]{{\fontsize{#1}{6}\selectfont#2 \par}}		%Cambia el size de todo el parrafo
\newcommand{\setlinesize}[2]{{\fontsize{#1}{6}\selectfont#2}}				%Cambia el font de una oración

\newcommand{\note}[1]{
	\begin{center}
		\huge{ \textcolor{red}{#1} }
	\end{center}
}

%FONTS (IMPORTANTE): Compilar en XeLaTex o LuaLaTeX
\usepackage{anyfontsize}	%Font size
\usepackage{fontspec}		%Font type

\usepackage{etoolbox}
\usepackage{todonotes}

\newcommand{\observacion}[2]{  \ifnumequal{1}{#1}{ { \todo[inline,backgroundcolor=red!25,bordercolor=red!100]{\textbf{Observación: #2}} } }{  }  }

\setcounter{topnumber}{2}
\setcounter{bottomnumber}{2}
\setcounter{totalnumber}{4}
\renewcommand{\topfraction}{0.85}
\renewcommand{\bottomfraction}{0.85}
\renewcommand{\textfraction}{0.15}
\renewcommand{\floatpagefraction}{0.8}
\renewcommand{\textfraction}{0.1}
\setlength{\floatsep}{5pt plus 2pt minus 2pt}
\setlength{\textfloatsep}{5pt plus 2pt minus 2pt}
\setlength{\intextsep}{5pt plus 2pt minus 2pt}

\newcommand{\quotes}[1]{``#1''}
\usepackage{array}
\newcolumntype{C}[1]{>{\centering\let\newline\\\arraybackslash\hspace{0pt}}m{#1}}
\usepackage[american]{circuitikz}
\usetikzlibrary{calc}
\usepackage{fancyhdr}
\usepackage{units} 

\graphicspath{{../Control de posición no lineal/}{../Control de fuerza no lineal/}{../Control híbrido no lineal/}{../Referencias/}{../Deducción de modelo/}{../Conclusiones/}}

\pagestyle{fancy}
\fancyhf{}
\lhead{22.99 - Automación Industrial}
\rhead{Lambertucci, Londero B., Maselli, Mechoulam}
\rfoot{Página \thepage}

%Items con bullets y no cuadrados
\renewcommand{\labelitemi}{\textbullet }

%%
%\begin{document}

En este trabajo fue posible modelar una planta física a través de la formulación de Euler-Lagrange y la linealización del mismo. También se definió la observabilidad y controlabilidad del sistema, tanto mediante cuentas analíticas como por inspección de diagramas de Mason. En el caso de estos, se optó por realizarlos para los tres casos estudiados, siendo estos: el sistema con fricción midiendo posición y ángulos de los joints; el sistema midiendo únicamente la posición; y el caso en el que se mide únicamente la posición y no hay fricción.
Adicionalmente se diseñaron lazos de control por realimentación de estados, tanto para el caso continuo como para el caso discreto discreto y también para el caso del sistema ampliado para realizar un control integral. Cabe mencionar que también se implementaron observadores para dichos sistemas exceptuando aquel con control integral.


Se aprendió aun más acerca del paquete Simscape y Simulink. Con ayuda de este, se pudo graficar la planta de manera dinámica y estética, realizar una comparativa entre los resultados de las diversas plantas y la velocidad con las que estas alcanzan a la referencia y analizar el error que presenta el observador comparado con el sistema realimentado idealmente.

Finalmente se realizó una pequeña comparativa entre el problema del doble péndulo invertido con carrito y el pédulo invertido con carrito simple.

PENDIENTE
%\end{document}