%\documentclass[a4paper]{article}
\usepackage[utf8]{inputenc}
\usepackage[spanish, es-tabla, es-noshorthands]{babel}
\usepackage[table,xcdraw]{xcolor}
\usepackage[a4paper, footnotesep = 1cm, width=22cm, top=2.5cm, height=25cm, textwidth=20cm, textheight=25cm]{geometry}
%\geometry{showframe}

\usepackage{tikz}
\usepackage{amsmath}
\usepackage{amsfonts}
\usepackage{amssymb}
\usepackage{float}
\usepackage{graphicx}
\usepackage{caption}
\usepackage{subcaption}
\usepackage{multicol}
\usepackage{multirow}
\usepackage{wrapfig}
\setlength{\doublerulesep}{\arrayrulewidth}
\usepackage{booktabs}

\usepackage{hyperref}
\hypersetup{
    colorlinks=true,
    linkcolor=blue,
    filecolor=magenta,      
    urlcolor=blue,
    citecolor=blue,    
}

\newcommand{\note}[1]{
	\begin{center}
		\huge{ \textcolor{red}{#1} }
	\end{center}
}

\setcounter{topnumber}{2}
\setcounter{bottomnumber}{2}
\setcounter{totalnumber}{4}
\renewcommand{\topfraction}{0.85}
\renewcommand{\bottomfraction}{0.85}
\renewcommand{\textfraction}{0.15}
\renewcommand{\floatpagefraction}{0.8}
\renewcommand{\textfraction}{0.1}
\setlength{\floatsep}{5pt plus 2pt minus 2pt}
\setlength{\textfloatsep}{5pt plus 2pt minus 2pt}
\setlength{\intextsep}{5pt plus 2pt minus 2pt}

\newcommand{\quotes}[1]{``#1''}
\usepackage{array}
\newcolumntype{C}[1]{>{\centering\let\newline\\\arraybackslash\hspace{0pt}}m{#1}}
\usepackage[american]{circuitikz}
\usetikzlibrary{calc}
\usepackage{fancyhdr}
\usepackage{units} 

\graphicspath{{../Ejercicio-1/}{../Ejercicio-2/}{../Ejercicio-3/}{../Ejercicio-4/}}

\pagestyle{fancy}
\fancyhf{}
\lhead{22.14 - Electrónica IV}
\rhead{Mechoulam, Lambertucci, Londero}
\rfoot{Página \thepage}

%
%\begin{document}

\begin{table}[H]
\centering
\begin{tabular}{!{\color{AzulTable}\vrule}ll!{\color{AzulTable}\vrule}}
\arrayrulecolor{AzulTable}
\hline
\rowcolor{AzulTable}
\multicolumn{1}{|l}{\textcolor{white}{Término}} & \multicolumn{1}{l|}{\textcolor{white}{Descripci\'on}} \\ \hline
\textbf{$x_0$}		& \begin{tabular}[l]{@{}l@{}}Posición horizontal del carrito.\end{tabular}							\\ \hline
\textbf{$\theta_1$}		& \begin{tabular}[l]{@{}l@{}}Ángulo entre el primer link con la vertical.\end{tabular}						\\ \hline
\textbf{$\theta_2$}		& \begin{tabular}[l]{@{}l@{}}Ángulo entre el segundo link con la vertical.\end{tabular}						\\ \hline
\textbf{$m_0$}		& \begin{tabular}[l]{@{}l@{}}Masa del carrito.\end{tabular}						\\ \hline
\textbf{$m_1$}		& \begin{tabular}[l]{@{}l@{}}Masa del primer link del péndulo.\end{tabular}						\\ \hline
\textbf{$m_2$}		& \begin{tabular}[l]{@{}l@{}}Masa del segundo link del péndulo..\end{tabular}						\\ \hline
\textbf{$L_1$}		& \begin{tabular}[l]{@{}l@{}}Longitud del primer link.\end{tabular}						\\ \hline
\textbf{$L_2$}		& \begin{tabular}[l]{@{}l@{}}Longitud del segundo link..\end{tabular}						\\ \hline

\textbf{$l_1$}		& \begin{tabular}[l]{@{}l@{}}Distancia entre la base del primer link y su centro de masas.\end{tabular}						\\ \hline

\textbf{$l_2$}		& \begin{tabular}[l]{@{}l@{}}
Distancia entre la base del segundo link y su centro de masas.\end{tabular}						\\ \hline

\textbf{$I_1$}		& \begin{tabular}[l]{@{}l@{}}
Momento de inercia del primer link.
\end{tabular}						\\ \hline

\textbf{$I_2$}		& \begin{tabular}[l]{@{}l@{}}
Momento de inercia del segundo link.
\end{tabular}						\\ \hline

\textbf{$g$}		& \begin{tabular}[l]{@{}l@{}}
Constante de aceleración de la gravedad.
\end{tabular}						\\ \hline

\textbf{$u(t)$}		& \begin{tabular}[l]{@{}l@{}}Acción de Control.\end{tabular}						\\ \hline
\textbf{$\Theta$}		& \begin{tabular}[l]{@{}l@{}}Vector de estados.\end{tabular}						\\ \hline
\textbf{$Y$}		& \begin{tabular}[l]{@{}l@{}}Vector de salida del sistema.\end{tabular}						
\\ \hline
\textbf{$\hat{\Theta}$}		& \begin{tabular}[l]{@{}l@{}}Vector de estados estimado.\end{tabular}						\\ \hline
\textbf{$\hat{Y}$}		& \begin{tabular}[l]{@{}l@{}}Vector de salida del sistema estimado.\end{tabular}						
\\ \hline

\textbf{$\Theta_0$}		& \begin{tabular}[l]{@{}l@{}}Condición Inicial.\end{tabular}						\\ \hline
\textbf{$L$}		& \begin{tabular}[l]{@{}l@{}}Lagrangiano.\end{tabular}						\\ \hline

\textbf{$M\left( \Theta \right)$}		& \begin{tabular}[l]{@{}l@{}}Matriz de influencia de masas/inercia en vector de torques.\end{tabular}						\\ \hline


\textbf{$V\left( \Theta , \dot{\Theta} \right)$}		& \begin{tabular}[l]{@{}l@{}}Vector de influencia de términos centrífugos y de Coriolis.\end{tabular}						\\ \hline

\textbf{$G\left( \Theta \right)$}		& \begin{tabular}[l]{@{}l@{}}Vector de influencia de la gravedad.\end{tabular}						\\ \hline


\end{tabular}
\end{table}
%\end{document}