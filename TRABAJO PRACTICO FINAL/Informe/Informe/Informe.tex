\documentclass[a4paper]{article}
\usepackage[utf8]{inputenc}
\usepackage[spanish, es-tabla, es-noshorthands]{babel}
\usepackage[table,xcdraw]{xcolor}
\usepackage[a4paper, footnotesep = 1cm, width=22cm, top=2.5cm, height=25cm, textwidth=20cm, textheight=25cm]{geometry}
%\geometry{showframe}

\usepackage{tikz}
\usepackage{amsmath}
\usepackage{amsfonts}
\usepackage{amssymb}
\usepackage{float}
\usepackage{graphicx}
\usepackage{caption}
\usepackage{subcaption}
\usepackage{multicol}
\usepackage{multirow}
\usepackage{wrapfig}
\setlength{\doublerulesep}{\arrayrulewidth}
\usepackage{booktabs}

\usepackage{hyperref}
\hypersetup{
    colorlinks=true,
    linkcolor=blue,
    filecolor=magenta,      
    urlcolor=blue,
    citecolor=blue,    
}

\newcommand{\note}[1]{
	\begin{center}
		\huge{ \textcolor{red}{#1} }
	\end{center}
}

\setcounter{topnumber}{2}
\setcounter{bottomnumber}{2}
\setcounter{totalnumber}{4}
\renewcommand{\topfraction}{0.85}
\renewcommand{\bottomfraction}{0.85}
\renewcommand{\textfraction}{0.15}
\renewcommand{\floatpagefraction}{0.8}
\renewcommand{\textfraction}{0.1}
\setlength{\floatsep}{5pt plus 2pt minus 2pt}
\setlength{\textfloatsep}{5pt plus 2pt minus 2pt}
\setlength{\intextsep}{5pt plus 2pt minus 2pt}

\newcommand{\quotes}[1]{``#1''}
\usepackage{array}
\newcolumntype{C}[1]{>{\centering\let\newline\\\arraybackslash\hspace{0pt}}m{#1}}
\usepackage[american]{circuitikz}
\usetikzlibrary{calc}
\usepackage{fancyhdr}
\usepackage{units} 

\graphicspath{{../Ejercicio-1/}{../Ejercicio-2/}{../Ejercicio-3/}{../Ejercicio-4/}}

\pagestyle{fancy}
\fancyhf{}
\lhead{22.14 - Electrónica IV}
\rhead{Mechoulam, Lambertucci, Londero}
\rfoot{Página \thepage}


\begin{document}

\def\verObs{0}

%%%%%%%%%%%%%%%%%%%%%%%%%
%		Caratula		%
%%%%%%%%%%%%%%%%%%%%%%%%%

\setmainfont{AvenirLTStd-Roman}[Path = ./Utils/, Extension = .otf]

\begin{titlepage}

\begin{tikzpicture}[remember picture, overlay, black, line width = 0.5pt]
	\coordinate (a) at (-2cm,2cm);
	\coordinate (b) at (17cm,-25.5cm);
	
	\coordinate (ap) at (-2.1cm,2.1cm);
	\coordinate (bp) at (17.1cm,-25.6cm);
	
	\draw[] (a) -| (b);
	\draw[] (a) |- (b);
	
	\draw[] (ap) -| (bp);
	\draw[] (ap) |- (bp);
	
	%footnotesep=1.25cm, headheight=1.25cm, top=2.54cm, left=2.54cm, bottom=2.54cm, right=2.54cm

\end{tikzpicture}

\begin{figure}[H]
	\includegraphics[width=0.3\linewidth, right]{./Utils/ITBA_1}
\end{figure}

\vspace*{0.5cm}

\noindent \textbf{\setlinesize{12}{INSTITUTO TECNOLÓGICO DE BUENOS AIRES - ITBA}}

\noindent \textbf{\setlinesize{12}{ESCUELA DE INGENIERÍA Y TECNOLOGÍA}}

\vspace*{4cm}

\begin{center}
	\setlinesize{24}{ \textbf{TRABAJO PRÁCTICO FINAL} }
	
	\vspace*{1.5cm}
	%\setlinesize{24}{ \textbf{Subtítulo del trabajo (cuando corresponda)} }
	\vspace*{1.0cm}
\end{center}
\begin{center}
	\setlinesize{18}{ \textbf{MANUAL DE USUARIO} }
	
	\vspace*{1.5cm}
	%\setlinesize{24}{ \textbf{Subtítulo del trabajo (cuando corresponda)} }
	\vspace*{1.0cm}
\end{center}
\begin{figure}[H]
\begin{adjustwidth}{-1cm}{}
\begin{tabular}{llr} 
	\textbf{AUTORES:}
	& \textbf{Lambertucci, Guido Enrique} & \textbf{(Leg. N}$\mathbf{^o}$ \textbf{58009)} \\
	& \textbf{Londero Bonaparte, Tomás Guillermo} & \textbf{(Leg. N}$\mathbf{^o}$ \textbf{58150)} \\
	& \textbf{Mechoulam, Alan}  &  \textbf{(Leg. N}$\mathbf{^o}$ \textbf{58438)}\\
	& \textbf{Maselli, Carlos Javier} &  \textbf{(Leg. N}$\mathbf{^o}$ \textbf{59564)} \\
	 
 &  & \\
 &  & \\
	\textbf{DOCENTES}:
	& \textbf{Arias, Rodolfo Enrique} & \\
	& \textbf{Sofio Avogadro, Federico} & \\
	& \textbf{Spinelli, Mariano Tomás} & \\
\end{tabular}
\end{adjustwidth}
\end{figure}

\vspace*{0.5cm}
\center{
%{\noindent \setparagraphsize{12}{\textbf{TRABAJO PRÁCTICO N$^{\circ}$1}}}
{\noindent \setparagraphsize{12}{\textbf{\textsc{22.90 - Automación Industrial}}}}
}
\vspace*{0.5cm}

\center{\textbf{BUENOS AIRES}}

\end{titlepage}


\setmainfont{Calibri}

%%%%%%%%%%%%%%%%%%%%%
%		Indice		%
%%%%%%%%%%%%%%%%%%%%%


%%%%%%%%%%%%%%%%%%%%%
%		Informe		%
%%%%%%%%%%%%%%%%%%%%%

\Aection{Introducción}
En el presente se explica que cosas son necesarias para correr el trabajo, asi mismo tambien como correrlo y utlizarlo.
\Section{Dependencias}
Las dependencias necesarias son:
\begin{itemize}
\item \href{https://petercorke.com/toolboxes/robotics-toolbox/}{Peter Corke Robotics Toolbox}
\item \href{https://petercorke.com/toolboxes/machine-vision-toolbox/}{Peter Corke Vision Toolbox}
\item \href{https://la.mathworks.com/products/matlab.html}{Matlab 2018}
\end{itemize}
\Section{Correr el trabajo}
Para poder correr el trabajo satisfactoriamente se debe abrir la carpeta "Widow" el archivo llamado "main.m" en matlab 2018.
Luego se da al correr, y se abrirá la siguiente interfaz:
\begin{figure}[H]
	\centering
	\includegraphics[width=0.8\linewidth]{GUI}
	\caption{GUI.}	
	\label{fig:GUI}
\end{figure}
\end{document}