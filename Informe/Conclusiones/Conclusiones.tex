%\documentclass[a4paper]{article}
\usepackage[utf8]{inputenc}
\usepackage[spanish, es-tabla, es-noshorthands]{babel}
\usepackage[table,xcdraw]{xcolor}
\usepackage[a4paper, footnotesep = 1cm, width=22cm, top=2.5cm, height=25cm, textwidth=20cm, textheight=25cm]{geometry}
%\geometry{showframe}

\usepackage{tikz}
\usepackage{amsmath}
\usepackage{amsfonts}
\usepackage{amssymb}
\usepackage{float}
\usepackage{graphicx}
\usepackage{caption}
\usepackage{subcaption}
\usepackage{multicol}
\usepackage{multirow}
\usepackage{wrapfig}
\setlength{\doublerulesep}{\arrayrulewidth}
\usepackage{booktabs}

\usepackage{hyperref}
\hypersetup{
    colorlinks=true,
    linkcolor=blue,
    filecolor=magenta,      
    urlcolor=blue,
    citecolor=blue,    
}

\newcommand{\note}[1]{
	\begin{center}
		\huge{ \textcolor{red}{#1} }
	\end{center}
}

\setcounter{topnumber}{2}
\setcounter{bottomnumber}{2}
\setcounter{totalnumber}{4}
\renewcommand{\topfraction}{0.85}
\renewcommand{\bottomfraction}{0.85}
\renewcommand{\textfraction}{0.15}
\renewcommand{\floatpagefraction}{0.8}
\renewcommand{\textfraction}{0.1}
\setlength{\floatsep}{5pt plus 2pt minus 2pt}
\setlength{\textfloatsep}{5pt plus 2pt minus 2pt}
\setlength{\intextsep}{5pt plus 2pt minus 2pt}

\newcommand{\quotes}[1]{``#1''}
\usepackage{array}
\newcolumntype{C}[1]{>{\centering\let\newline\\\arraybackslash\hspace{0pt}}m{#1}}
\usepackage[american]{circuitikz}
\usetikzlibrary{calc}
\usepackage{fancyhdr}
\usepackage{units} 

\graphicspath{{../Ejercicio-1/}{../Ejercicio-2/}{../Ejercicio-3/}{../Ejercicio-4/}}

\pagestyle{fancy}
\fancyhf{}
\lhead{22.14 - Electrónica IV}
\rhead{Mechoulam, Lambertucci, Londero}
\rfoot{Página \thepage}

%
%\begin{document}

Se tuvo la oportunidad de desarrollar analíticamente la mecánica del manipulador propuesto por la cátedra, obteniendo a través del método de Lagrange el vector de torques, al igual que por propagación de velocidades la matriz jacobiana. Se llevaron adelante diversos tipos de control, tanto de posición y fuerzo e incluso uno híbrido, el cual incluye ambas variables mencioandas.

Se exploraron varias topologías de control no lineal, como puede ser la linealización por punto de equilibrio variable y la linealización por realimentación. La utilizada fue la linealización por realimentación debido al amplio conocimiento que se tiene sobre la planta. 

Se profundizó en el uso de simulink, al igual que un aprendizaje en el uso del toolbox de robotics de Peter Corke, herramienta para la simulación de manipuladores. 

Finalmente se observó la reacción de los distintos tipos de control ante disturbios en la planta, y como estos vuelven a sus señales respectivas de referencia.

%\end{document}