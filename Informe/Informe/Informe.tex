\documentclass[a4paper]{article}
\usepackage[utf8]{inputenc}
\usepackage[spanish, es-tabla, es-noshorthands]{babel}
\usepackage[table,xcdraw]{xcolor}
\usepackage[a4paper, footnotesep = 1cm, width=22cm, top=2.5cm, height=25cm, textwidth=20cm, textheight=25cm]{geometry}
%\geometry{showframe}

\usepackage{tikz}
\usepackage{amsmath}
\usepackage{amsfonts}
\usepackage{amssymb}
\usepackage{float}
\usepackage{graphicx}
\usepackage{caption}
\usepackage{subcaption}
\usepackage{multicol}
\usepackage{multirow}
\usepackage{wrapfig}
\setlength{\doublerulesep}{\arrayrulewidth}
\usepackage{booktabs}

\usepackage{hyperref}
\hypersetup{
    colorlinks=true,
    linkcolor=blue,
    filecolor=magenta,      
    urlcolor=blue,
    citecolor=blue,    
}

\newcommand{\note}[1]{
	\begin{center}
		\huge{ \textcolor{red}{#1} }
	\end{center}
}

\setcounter{topnumber}{2}
\setcounter{bottomnumber}{2}
\setcounter{totalnumber}{4}
\renewcommand{\topfraction}{0.85}
\renewcommand{\bottomfraction}{0.85}
\renewcommand{\textfraction}{0.15}
\renewcommand{\floatpagefraction}{0.8}
\renewcommand{\textfraction}{0.1}
\setlength{\floatsep}{5pt plus 2pt minus 2pt}
\setlength{\textfloatsep}{5pt plus 2pt minus 2pt}
\setlength{\intextsep}{5pt plus 2pt minus 2pt}

\newcommand{\quotes}[1]{``#1''}
\usepackage{array}
\newcolumntype{C}[1]{>{\centering\let\newline\\\arraybackslash\hspace{0pt}}m{#1}}
\usepackage[american]{circuitikz}
\usetikzlibrary{calc}
\usepackage{fancyhdr}
\usepackage{units} 

\graphicspath{{../Ejercicio-1/}{../Ejercicio-2/}{../Ejercicio-3/}{../Ejercicio-4/}}

\pagestyle{fancy}
\fancyhf{}
\lhead{22.14 - Electrónica IV}
\rhead{Mechoulam, Lambertucci, Londero}
\rfoot{Página \thepage}


\begin{document}

%%%%%%%%%%%%%%%%%%%%%%%%%
%		Caratula		%
%%%%%%%%%%%%%%%%%%%%%%%%%

%\setmainfont{Avenir LT Std 55 Roman}
\setmainfont{AvenirLTStd-Roman}

\begin{titlepage}

\begin{tikzpicture}[remember picture, overlay, black, line width = 0.5pt]
	\coordinate (a) at (-2cm,2cm);
	\coordinate (b) at (17cm,-25.5cm);
	
	\coordinate (ap) at (-2.1cm,2.1cm);
	\coordinate (bp) at (17.1cm,-25.6cm);
	
	\draw[] (a) -| (b);
	\draw[] (a) |- (b);
	
	\draw[] (ap) -| (bp);
	\draw[] (ap) |- (bp);
	
	%footnotesep=1.25cm, headheight=1.25cm, top=2.54cm, left=2.54cm, bottom=2.54cm, right=2.54cm

\end{tikzpicture}

\begin{figure}[H]
	\includegraphics[width=0.3\linewidth, right]{./Utils/ITBA_1}
\end{figure}

\vspace*{0.5cm}

\noindent \textbf{\setlinesize{12}{INSTITUTO TECNOLÓGICO DE BUENOS AIRES - ITBA}}

\noindent \textbf{\setlinesize{12}{ESCUELA DE INGENIERÍA Y TECNOLOGÍA}}

\vspace*{4cm}

\begin{center}
	\setlinesize{24}{ \textbf{TRABAJO PRÁCTICO FINAL} }
	
	\vspace*{1.5cm}
	%\setlinesize{24}{ \textbf{Subtítulo del trabajo (cuando corresponda)} }
	\vspace*{1.0cm}
\end{center}
\begin{center}
	\setlinesize{18}{ \textbf{MANUAL DE USUARIO} }
	
	\vspace*{1.5cm}
	%\setlinesize{24}{ \textbf{Subtítulo del trabajo (cuando corresponda)} }
	\vspace*{1.0cm}
\end{center}
\begin{figure}[H]
\begin{adjustwidth}{-1cm}{}
\begin{tabular}{llr} 
	\textbf{AUTORES:}
	& \textbf{Lambertucci, Guido Enrique} & \textbf{(Leg. N}$\mathbf{^o}$ \textbf{58009)} \\
	& \textbf{Londero Bonaparte, Tomás Guillermo} & \textbf{(Leg. N}$\mathbf{^o}$ \textbf{58150)} \\
	& \textbf{Mechoulam, Alan}  &  \textbf{(Leg. N}$\mathbf{^o}$ \textbf{58438)}\\
	& \textbf{Maselli, Carlos Javier} &  \textbf{(Leg. N}$\mathbf{^o}$ \textbf{59564)} \\
	 
 &  & \\
 &  & \\
	\textbf{DOCENTES}:
	& \textbf{Arias, Rodolfo Enrique} & \\
	& \textbf{Sofio Avogadro, Federico} & \\
	& \textbf{Spinelli, Mariano Tomás} & \\
\end{tabular}
\end{adjustwidth}
\end{figure}

\vspace*{0.5cm}
\center{
%{\noindent \setparagraphsize{12}{\textbf{TRABAJO PRÁCTICO N$^{\circ}$1}}}
{\noindent \setparagraphsize{12}{\textbf{\textsc{22.90 - Automación Industrial}}}}
}
\vspace*{0.5cm}

\center{\textbf{BUENOS AIRES}}

\end{titlepage}


\setmainfont{Calibri}

%%%%%%%%%%%%%%%%%%%%%
%		Indice		%
%%%%%%%%%%%%%%%%%%%%%

\tableofcontents
\newpage

%%%%%%%%%%%%%%%%%%%%%
%		Informe		%
%%%%%%%%%%%%%%%%%%%%%

\section{Deducción de modelo}
%\documentclass[a4paper]{article}
\usepackage[utf8]{inputenc}
\usepackage[spanish, es-tabla, es-noshorthands]{babel}
\usepackage[table,xcdraw]{xcolor}
\usepackage[a4paper, footnotesep = 1cm, width=22cm, top=2.5cm, height=25cm, textwidth=20cm, textheight=25cm]{geometry}
%\geometry{showframe}

\usepackage{tikz}
\usepackage{amsmath}
\usepackage{amsfonts}
\usepackage{amssymb}
\usepackage{float}
\usepackage{graphicx}
\usepackage{caption}
\usepackage{subcaption}
\usepackage{multicol}
\usepackage{multirow}
\usepackage{wrapfig}
\setlength{\doublerulesep}{\arrayrulewidth}
\usepackage{booktabs}

\usepackage{hyperref}
\hypersetup{
    colorlinks=true,
    linkcolor=blue,
    filecolor=magenta,      
    urlcolor=blue,
    citecolor=blue,    
}

\newcommand{\note}[1]{
	\begin{center}
		\huge{ \textcolor{red}{#1} }
	\end{center}
}

\setcounter{topnumber}{2}
\setcounter{bottomnumber}{2}
\setcounter{totalnumber}{4}
\renewcommand{\topfraction}{0.85}
\renewcommand{\bottomfraction}{0.85}
\renewcommand{\textfraction}{0.15}
\renewcommand{\floatpagefraction}{0.8}
\renewcommand{\textfraction}{0.1}
\setlength{\floatsep}{5pt plus 2pt minus 2pt}
\setlength{\textfloatsep}{5pt plus 2pt minus 2pt}
\setlength{\intextsep}{5pt plus 2pt minus 2pt}

\newcommand{\quotes}[1]{``#1''}
\usepackage{array}
\newcolumntype{C}[1]{>{\centering\let\newline\\\arraybackslash\hspace{0pt}}m{#1}}
\usepackage[american]{circuitikz}
\usetikzlibrary{calc}
\usepackage{fancyhdr}
\usepackage{units} 

\graphicspath{{../Ejercicio-1/}{../Ejercicio-2/}{../Ejercicio-3/}{../Ejercicio-4/}}

\pagestyle{fancy}
\fancyhf{}
\lhead{22.14 - Electrónica IV}
\rhead{Mechoulam, Lambertucci, Londero}
\rfoot{Página \thepage}


%\begin{document}

La consigna propone un manipulador RR de las siguientes cualidades.
\begin{figure}[H]
	\centering
	\includegraphics[width=0.8\linewidth]{ImagenesDeducción de modelo/brazo}
	\caption{Manipulador RR.}	
	\label{fig:brazo}
\end{figure}

Para los parámetros DH se optó por la posición $\theta_1 = 0$ y $\theta_2 = 0$.
Planteando todas las ternas paralelas, es decir, los ejes Z paralelos, los X colineales, se obtuvieron los siguientes parámetros.
\begin{table}[H]
\centering
\begin{tabular}{ccccc}
\hline
\textbf{}   & \textbf{$\alpha$} & \textbf{a} & \textbf{$\theta$} & \textbf{d} \\ \hline
\textbf{1}  & 0                 & 0          & $\theta_1$        & 0          \\
\textbf{2}  & 0                 & $L_1$      & $\theta_2$        & 0          \\
\textbf{EE} & 0                 & $L_2$      & 0                 & 0          \\ \hline
\end{tabular}
\end{table}

Luego realizando la propagación de velocidades se obtiene que:
\begin{equation}
^1v_1 = 0 
\end{equation}
\begin{equation}
^1\omega_1= \dot{\theta_1} \ \cdot \ \hat{k}
\end{equation}
\begin{equation}
^2v_2 = \dot{\theta_1} \cdot \sin(\theta_2)L \  \cdot \ \hat{i} + \dot{\theta_1} \cdot \cos(\theta_2)L  \ \cdot \  \hat{j}
\end{equation}
\begin{equation}
^2\omega_2= \dot{\theta_1}+\dot{\theta_2}  \ \cdot \ \hat{k}
\end{equation}

A continuación, con los correspondientes a los centros de masa, ubicados al final de cada link, se calculan las velocidades de estos.
\begin{equation*}
^1v_{c1}=\dot{\theta_1}L \ \cdot \  \hat{j}
\end{equation*}
\begin{equation*}
^2v_{c2}=\dot{\theta_1} \cdot \sin(\theta_2)L  \ \cdot \ \hat{i} + \left( \dot{\theta_1} \cdot \cos(\theta_2)L  + L( \dot{\theta_1}+\dot{\theta_2} ) \right) \ \cdot \  \hat{j}
\end{equation*}

Además las matrices de inercia son diagonales con valores $I_{zz}=mL^2$, $I_{yy}=mL^2$ e $I_{xx}=0$.

Luego se procede a calcular el vector de torques.
\begin{equation}
\mathcal{L}(\Theta , \dot{\Theta}) = k(\Theta , \dot{\Theta}) - u(\Theta)
\end{equation}
\begin{equation*}
\tau = \frac{d}{dt}\left(\frac{\partial k}{\partial \dot{\Theta}} \right) - \frac{\partial k}{\partial \Theta} + \frac{\partial u}{\partial \Theta}
\end{equation*}

Debido a que todo el movimiento del brazo se encuentra al mismo potencial gravitatorio los términos de $u$ son nulos.

Operando se obtiene un modelo de la siguiente forma:
\begin{equation*}
\tau =  M(\Theta) \ddot{\Theta} + V(\Theta , \dot{\Theta}) + G(\Theta) + F(\Theta , \dot{\Theta})
\end{equation*}
\begin{equation}
\tau = \begin{pmatrix}
2m_2L^2+2I_{zz}+m_1L^2 & m_2L^2+I_{zz}\\
m_2L^2+I_{zz} & m_2L^2+I_{zz}
\end{pmatrix} 
\begin{pmatrix}
\ddot{\theta_1} \\ 
\ddot{\theta_2}
\end{pmatrix}
+
\begin{pmatrix}
-\dot{\theta_1}b_1-L\sin(\theta_2)m_2\dot{\theta_2} \\ 
-\dot{\theta_2}b_2+L\sin(\theta_2)m_2\dot{\theta_1}
\end{pmatrix}
\end{equation}

Adicionalmente se obtuvo el Jacobiano. Si bien este es una matriz de 3x2, dado a que nunca hay un movimiento en el eje Z, se toma el sistema de 2x2.
\begin{equation}
^{EE}J=\begin{pmatrix}
L\sin(\theta_2) & 0 \\
L\left[\cos(\theta+2)+1\right] & L
\end{pmatrix}
\end{equation}
%\end{document}

\section{Control de posición no lineal}
%\documentclass[a4paper]{article}
\usepackage[utf8]{inputenc}
\usepackage[spanish, es-tabla, es-noshorthands]{babel}
\usepackage[table,xcdraw]{xcolor}
\usepackage[a4paper, footnotesep = 1cm, width=22cm, top=2.5cm, height=25cm, textwidth=20cm, textheight=25cm]{geometry}
%\geometry{showframe}

\usepackage{tikz}
\usepackage{amsmath}
\usepackage{amsfonts}
\usepackage{amssymb}
\usepackage{float}
\usepackage{graphicx}
\usepackage{caption}
\usepackage{subcaption}
\usepackage{multicol}
\usepackage{multirow}
\usepackage{wrapfig}
\setlength{\doublerulesep}{\arrayrulewidth}
\usepackage{booktabs}

\usepackage{hyperref}
\hypersetup{
    colorlinks=true,
    linkcolor=blue,
    filecolor=magenta,      
    urlcolor=blue,
    citecolor=blue,    
}

\newcommand{\note}[1]{
	\begin{center}
		\huge{ \textcolor{red}{#1} }
	\end{center}
}

\setcounter{topnumber}{2}
\setcounter{bottomnumber}{2}
\setcounter{totalnumber}{4}
\renewcommand{\topfraction}{0.85}
\renewcommand{\bottomfraction}{0.85}
\renewcommand{\textfraction}{0.15}
\renewcommand{\floatpagefraction}{0.8}
\renewcommand{\textfraction}{0.1}
\setlength{\floatsep}{5pt plus 2pt minus 2pt}
\setlength{\textfloatsep}{5pt plus 2pt minus 2pt}
\setlength{\intextsep}{5pt plus 2pt minus 2pt}

\newcommand{\quotes}[1]{``#1''}
\usepackage{array}
\newcolumntype{C}[1]{>{\centering\let\newline\\\arraybackslash\hspace{0pt}}m{#1}}
\usepackage[american]{circuitikz}
\usetikzlibrary{calc}
\usepackage{fancyhdr}
\usepackage{units} 

\graphicspath{{../Ejercicio-1/}{../Ejercicio-2/}{../Ejercicio-3/}{../Ejercicio-4/}}

\pagestyle{fancy}
\fancyhf{}
\lhead{22.14 - Electrónica IV}
\rhead{Mechoulam, Lambertucci, Londero}
\rfoot{Página \thepage}


%\begin{document}

\subsection{Caracterización del problema}
Se desarrolla un control cartesiano no lineal para el manipulador RR. Además se debe considerar una zona prohibida, representada por todo aquel valor por encima de una pared, descrita en el plano XY pro la siguiente ecuación:
\begin{equation}
y=2-x
\end{equation}

Se busca que el manipulador se desplace desde el punto (1;-1;0) al (1;1;0). Para generar la trayectoria se utiliza la función \textbf{jtraj} del toolbox de matlab de Peter Corke.

\subsection{Esquema de control}
El modelo de control propuesto es el conocido como linealización por realimentación. Es fundamental para este tipo de control tener un gran conocimiento de la planta, ya que básicamente se realiza el control como si fuese lineal, con un esquema tipo PD, con la diferencia que se agrega a la acción de control la respuesta no lineal de la planta.
%gracias al conocimiento del modelo no lineal de la planta y sus variables de estado.

\begin{figure}[H]
	\centering
	\includegraphics[width=0.8\linewidth]{ImagenesControl de posición no lineal/modelo_control_p}
	\caption{Topología del control de posición cartesiano no lineal.}	
	\label{fig:control_p_modelo}
\end{figure}

Cabe mencionar que las matrices $M_x,\ V_x, \ y \ G_x$ se encuentran en espacio cartesiano. La manera de pasar de las mismas en espacio joint es la siguiente:
\begin{equation*}
M_x(\Theta) = J^{-T}(\Theta) M(\Theta) J^{-1}(\Theta)
\end{equation*} 
\begin{equation*}
V_x(\Theta , \dot{\Theta}) = J^{-T}(\Theta) \left( V(\Theta , \dot{\Theta}) - M(\Theta) J^{-1}(\Theta) \dot{J}(\Theta) \dot{\Theta} \right)
\end{equation*} 
\begin{equation*}
G_x(\Theta) = J^{-T}(\Theta) G(\Theta) 
\end{equation*}

\observacion{\verObs}{HABLAR DE VALORES DE GANANCIAS}

Para obtener las ganancias del sistema se empleo el método de Ziegler-Nichols. De esta forma, los valores obtenidos fueron:
\begin{itemize}
	\item Kv = [80 0;0 80].
	\item Kp = [250 0; 0 250].
\end{itemize}

\subsection{Resultados}
Se realizó el sistema en simulink. Se obtuvieron los siguientes gráficos.
En estos se pueden observar los ángulos de los manipuladores en espacio de joint.

Como el primer rotacional hace una trayectoria de $-\frac{\pi}{2}$ hacia $0$, y el segundo, si bien el punto inicial y final son el mismo, se desvía con el propósito de seguir la trayectoria cartesiana indicada.

\begin{figure}[H]
	\centering
	\includegraphics[width=0.8\linewidth]{ImagenesControl de posición no lineal/1_3_a}
	\caption{Ángulos en función del tiempo en espacio joint.}	
	\label{fig:athetas}
\end{figure}


En la Figura (\ref{fig:apos}) se ven tanto las referencias como las coordenadas reales que tomó el EE, con un error porcentual menor del 20$\%$ en el eje X y menor del 5$\%$ en el eje Y. 

\observacion{\verObs}{Porcentaje desconocido.}

\begin{figure}[H]
	\centering
	\includegraphics[width=0.8\linewidth]{ImagenesControl de posición no lineal/1_3_b}
	\caption{Posición deseada y real del EE.}	
	\label{fig:apos}
\end{figure}
La trayectoria descripta por el EE se observa claramente en la siguiente imagen.
\begin{figure}[H]
	\centering
	\includegraphics[width=0.5\linewidth]{ImagenesControl de posición no lineal/1_3_c}
	\caption{Gráfico XY.}	
	\label{fig:axy}
\end{figure}

Ademas se le incluyó un disturbio a la planta tanto en posición como en velocidad. Este disturbio sucede en el segundo 14. Se observa en los siguientes gráficos como el manipulador es afectado por el mismo y luego vuelve rápidamente a la referencia.

\begin{figure}[H]
	\centering
	\includegraphics[width=0.8\linewidth]{ImagenesControl de posición no lineal/1_3_e_a}
	\caption{Ángulos en función del tiempo en espacio joint.}	
	\label{fig:athetasd}
\end{figure}

\begin{figure}[H]
	\centering
	\includegraphics[width=0.8\linewidth]{ImagenesControl de posición no lineal/1_3_e_b}
	\caption{Posición deseada y real del EE.}	
	\label{fig:aposd}
\end{figure}

\begin{figure}[H]
	\centering
	\includegraphics[width=0.5\linewidth]{ImagenesControl de posición no lineal/1_3_e_c}
	\caption{Gráfico XY.}	
	\label{fig:axyd}
\end{figure}

De manera similar al caso presentado en la Sección (\ref{sec:posic}), se utilizó el método de Ziegler-Nichols. De esta forma, los valores obtenidos fueron:
\begin{itemize}
	\item Kv = [150 0;0 150].
	\item Kp = [250 0; 0 250].
\end{itemize}

%\end{document}

\section{Control de fuerza no lineal}
%\documentclass[a4paper]{article}
\usepackage[utf8]{inputenc}
\usepackage[spanish, es-tabla, es-noshorthands]{babel}
\usepackage[table,xcdraw]{xcolor}
\usepackage[a4paper, footnotesep = 1cm, width=22cm, top=2.5cm, height=25cm, textwidth=20cm, textheight=25cm]{geometry}
%\geometry{showframe}

\usepackage{tikz}
\usepackage{amsmath}
\usepackage{amsfonts}
\usepackage{amssymb}
\usepackage{float}
\usepackage{graphicx}
\usepackage{caption}
\usepackage{subcaption}
\usepackage{multicol}
\usepackage{multirow}
\usepackage{wrapfig}
\setlength{\doublerulesep}{\arrayrulewidth}
\usepackage{booktabs}

\usepackage{hyperref}
\hypersetup{
    colorlinks=true,
    linkcolor=blue,
    filecolor=magenta,      
    urlcolor=blue,
    citecolor=blue,    
}

\newcommand{\note}[1]{
	\begin{center}
		\huge{ \textcolor{red}{#1} }
	\end{center}
}

\setcounter{topnumber}{2}
\setcounter{bottomnumber}{2}
\setcounter{totalnumber}{4}
\renewcommand{\topfraction}{0.85}
\renewcommand{\bottomfraction}{0.85}
\renewcommand{\textfraction}{0.15}
\renewcommand{\floatpagefraction}{0.8}
\renewcommand{\textfraction}{0.1}
\setlength{\floatsep}{5pt plus 2pt minus 2pt}
\setlength{\textfloatsep}{5pt plus 2pt minus 2pt}
\setlength{\intextsep}{5pt plus 2pt minus 2pt}

\newcommand{\quotes}[1]{``#1''}
\usepackage{array}
\newcolumntype{C}[1]{>{\centering\let\newline\\\arraybackslash\hspace{0pt}}m{#1}}
\usepackage[american]{circuitikz}
\usetikzlibrary{calc}
\usepackage{fancyhdr}
\usepackage{units} 

\graphicspath{{../Ejercicio-1/}{../Ejercicio-2/}{../Ejercicio-3/}{../Ejercicio-4/}}

\pagestyle{fancy}
\fancyhf{}
\lhead{22.14 - Electrónica IV}
\rhead{Mechoulam, Lambertucci, Londero}
\rfoot{Página \thepage}


%\begin{document}

\subsection{Caracterizaci\'on del problema}
Para este caso se pidi\'o un control de fuerzas no lineal. Para ello un aspecto fundamental es modelar las fuerzas del sistema que interaccionan con el actuador.
En este caso es la pared que trabaja como obst\'aculo. La fuerza de reacci\'on entre el EE y la pared ser\'a aquella definida por:
\begin{equation}
f_r = k_e \cdot d \ \hat{n}
\end{equation}
La cual es una fuerza proporcional a la distancia y normal a la superficie de la pared.
Esta fuerza  ser\'a nula mientras el EE no se encuentre en contacto con la pared y no nula en caso de estarlo.

\subsection{Esquema de control propuesto}
El esquema de control porpuesto ser\'a nuevamente una linealizaci\'on por realimentaci\'on de la siguiente manera.
\begin{figure}[H]
	\centering
	\includegraphics[width=0.8\linewidth]{ImagenesControl de fuerza no lineal/controlf}
	\caption{Topolog\'ia del control de fuerza no lineal.}	
	\label{fig:control_f_modelo}
\end{figure}


HABLAR DE VALROES DE GANANCIAS
\subsection{Resultados}
Se realiz\'o el simulink del sistema. Obteniendo los siguientes gr\'aficos.
Algo carácteristico que tiene este control de fuerzas es que debido a que no hay información provista como referencia de posición, el manipulador se moverá hasta estar cerca de la pared, y leugo realizará una breve oscilación sobre la pared hasta llegar a un estado permanente en el cual el manipulador estará apicando la fuerza comandada por la referencia.
\begin{figure}[H]
	\centering
	\includegraphics[width=0.8\linewidth]{ImagenesControl de fuerza no lineal/2_3_a}
	\caption{\'Angulos en funci\'on del tiempo en espacio joint.}	
	\label{fig:athetas}
\end{figure}
Se observa claramente en la siguiente imagen, como oscila levemente el EE al llegar a la pared.
\begin{figure}[H]
	\centering
	\includegraphics[width=0.8\linewidth]{ImagenesControl de fuerza no lineal/2_3_b}
	\caption{Posici\'on  del EE.}	
	\label{fig:apos}
\end{figure}
\begin{figure}[H]
	\centering
	\includegraphics[width=0.5\linewidth]{ImagenesControl de fuerza no lineal/2_3_c}
	\caption{Gr\'afico XY.}	
	\label{fig:axy}
\end{figure}
Cabe mencionar en este gráfico que en todos los puntos en los cuales la función es cero, se debe a que el manipulador no esta en contacto con la pared por lo que la misma no aplica una fuerza.
Además se ve como una vez alcanzada la pared y dada una referencia, el EE osicla levemente en torno a la referencia hasta establecerse.
\begin{figure}[H]
	\centering
	\includegraphics[width=0.8\linewidth]{ImagenesControl de fuerza no lineal/2_3_e}
	\caption{Gr\'afico XY.}	
	\label{fig:af}
\end{figure}
Ademas se le incluy\'o un disturbio a la planta tanto en posici\'on como en velocidad.
Se puede ver claramente en este gráfico el momento en el que se le aplica el disturbio a la planta, y como las variables de estado se estabilizan nuevamente.
\begin{figure}[H]
	\centering
	\includegraphics[width=0.8\linewidth]{ImagenesControl de fuerza no lineal/2_3_f_a}
	\caption{\'Angulos en funci\'on del tiempo en espacio joint.}	
	\label{fig:athetasd}
\end{figure}
Aqui se ve el movimiento del EE.
\begin{figure}[H]
	\centering
	\includegraphics[width=0.8\linewidth]{ImagenesControl de fuerza no lineal/2_3_f_b}
	\caption{Posici\'on del EE.}	
	\label{fig:aposd}
\end{figure}
\begin{figure}[H]
	\centering
	\includegraphics[width=0.5\linewidth]{ImagenesControl de fuerza no lineal/2_3_f_c}
	\caption{Gr\'afico XY.}	
	\label{fig:bxyd}
\end{figure}
Y finalmente se observa un gráfico similar al de fuerzas anterior, con la diferencia de que se hace nulo el valor de la fuerza aplicada en el segundo 15 al EE despegarse completamente de la pared. luego vuelve a la misma y se establece en el régimen permanente nuevamente.
\begin{figure}[H]
	\centering
	\includegraphics[width=0.8\linewidth]{ImagenesControl de fuerza no lineal/2_3_f_e}
	\caption{Fuerza deseada y real.}	
	\label{fig:bfd}
\end{figure}
%\end{document}


\section{Control híbrido no lineal}
%\documentclass[a4paper]{article}
\usepackage[utf8]{inputenc}
\usepackage[spanish, es-tabla, es-noshorthands]{babel}
\usepackage[table,xcdraw]{xcolor}
\usepackage[a4paper, footnotesep = 1cm, width=22cm, top=2.5cm, height=25cm, textwidth=20cm, textheight=25cm]{geometry}
%\geometry{showframe}

\usepackage{tikz}
\usepackage{amsmath}
\usepackage{amsfonts}
\usepackage{amssymb}
\usepackage{float}
\usepackage{graphicx}
\usepackage{caption}
\usepackage{subcaption}
\usepackage{multicol}
\usepackage{multirow}
\usepackage{wrapfig}
\setlength{\doublerulesep}{\arrayrulewidth}
\usepackage{booktabs}

\usepackage{hyperref}
\hypersetup{
    colorlinks=true,
    linkcolor=blue,
    filecolor=magenta,      
    urlcolor=blue,
    citecolor=blue,    
}

\newcommand{\note}[1]{
	\begin{center}
		\huge{ \textcolor{red}{#1} }
	\end{center}
}

\setcounter{topnumber}{2}
\setcounter{bottomnumber}{2}
\setcounter{totalnumber}{4}
\renewcommand{\topfraction}{0.85}
\renewcommand{\bottomfraction}{0.85}
\renewcommand{\textfraction}{0.15}
\renewcommand{\floatpagefraction}{0.8}
\renewcommand{\textfraction}{0.1}
\setlength{\floatsep}{5pt plus 2pt minus 2pt}
\setlength{\textfloatsep}{5pt plus 2pt minus 2pt}
\setlength{\intextsep}{5pt plus 2pt minus 2pt}

\newcommand{\quotes}[1]{``#1''}
\usepackage{array}
\newcolumntype{C}[1]{>{\centering\let\newline\\\arraybackslash\hspace{0pt}}m{#1}}
\usepackage[american]{circuitikz}
\usetikzlibrary{calc}
\usepackage{fancyhdr}
\usepackage{units} 

\graphicspath{{../Ejercicio-1/}{../Ejercicio-2/}{../Ejercicio-3/}{../Ejercicio-4/}}

\pagestyle{fancy}
\fancyhf{}
\lhead{22.14 - Electrónica IV}
\rhead{Mechoulam, Lambertucci, Londero}
\rfoot{Página \thepage}


%\begin{document}

\subsection{Caracterizaci\'on del problema}
Finalmente se realiz\'o un control hibrido, en el cual toman parte tanto posici\'on como fuerza.


\subsection{Esquema de control propuesto}
El esquema de control porpuesto ser\'a nuevamente una linealizaci\'on por realimentaci\'on de la siguiente manera.
\begin{figure}[H]
	\centering
	\includegraphics[width=0.8\linewidth]{ImagenesControl híbrido no lineal/controlh}
	\caption{Topolog\'ia del control de fuerza no lineal.}	
	\label{fig:control_f_modelo}
\end{figure}


HABLAR DE VALROES DE GANANCIAS
\subsection{Resultados}
Se realiz\'o el simulink del sistema. Obteniendo los siguientes gr\'aficos.

\begin{figure}[H]
	\centering
	\includegraphics[width=0.8\linewidth]{ImagenesControl híbrido no lineal/3_3_a}
	\caption{\'Angulos en funci\'on del tiempo en espacio joint.}	
	\label{fig:cthetas}
\end{figure}

\begin{figure}[H]
	\centering
	\includegraphics[width=0.8\linewidth]{ImagenesControl híbrido no lineal/3_3_b}
	\caption{Posici\'on  del EE.}	
	\label{fig:cpos}
\end{figure}
\begin{figure}[H]
	\centering
	\includegraphics[width=0.5\linewidth]{ImagenesControl híbrido no lineal/3_3_c}
	\caption{Gr\'afico XY.}	
	\label{fig:cxy}
\end{figure}
\begin{figure}[H]
	\centering
	\includegraphics[width=0.8\linewidth]{ImagenesControl híbrido no lineal/3_3_e}
	\caption{Gr\'afico XY.}	
	\label{fig:cf}
\end{figure}
Ademas se le incluy\'o un disturbio a la planta tanto en posici\'on como en velocidad.
\begin{figure}[H]
	\centering
	\includegraphics[width=0.8\linewidth]{ImagenesControl híbrido no lineal/3_3_f_a}
	\caption{\'Angulos en funci\'on del tiempo en espacio joint.}	
	\label{fig:cthetasd}
\end{figure}

\begin{figure}[H]
	\centering
	\includegraphics[width=0.8\linewidth]{ImagenesControl híbrido no lineal/3_3_f_b}
	\caption{Posici\'on del EE.}	
	\label{fig:cposd}
\end{figure}
\begin{figure}[H]
	\centering
	\includegraphics[width=0.5\linewidth]{ImagenesControl híbrido no lineal/3_3_f_c}
	\caption{Gr\'afico XY.}	
	\label{fig:cxyd}
\end{figure}
\begin{figure}[H]
	\centering
	\includegraphics[width=0.8\linewidth]{ImagenesControl híbrido no lineal/3_3_f_e}
	\caption{Fuerza deseada y real.}	
	\label{fig:cfd}
\end{figure}
%\end{document}

\section{Conclusiones}
Hubo la oportunidad de desarrollar analíticamente la mecánica del manipulador propuesto por la cátedra, obteniendo a través del método de Lagrange el vector de torques, al igual que por propagación de velocidades la matriz jacobiana. Parámetros sumamente importantes para la siguiente actividad, la cual fue llevar a cabo diversos tipos de control, tanto de posición y fuerzo e incluso uno híbrido el cual incluía tanto posición como fuerza.

 Se exploraron varias topologías de control no lineal, como puede ser la linealización por punto de equilibrio variable, y la linealización por realimentación. La utilizada fue la linealización por realimetnación debido al amplio conocimiento que se tiene sobre la planta. 

Se profundizó en el uso de simulink  al igual que un aprendizaje en el uso del toolbox de robotics de Peter Corke, una gran herramienta para la simulación de manipuladores. 

Finalmente se observó como era la reacción de los distintos tipos de control ante disturbios en la planta.
Y como estos volvían a sus señales respectivas de referencia.

%\newpage
%\documentclass[a4paper]{article}
\usepackage[utf8]{inputenc}
\usepackage[spanish, es-tabla, es-noshorthands]{babel}
\usepackage[table,xcdraw]{xcolor}
\usepackage[a4paper, footnotesep = 1cm, width=22cm, top=2.5cm, height=25cm, textwidth=20cm, textheight=25cm]{geometry}
%\geometry{showframe}

\usepackage{tikz}
\usepackage{amsmath}
\usepackage{amsfonts}
\usepackage{amssymb}
\usepackage{float}
\usepackage{graphicx}
\usepackage{caption}
\usepackage{subcaption}
\usepackage{multicol}
\usepackage{multirow}
\usepackage{wrapfig}
\setlength{\doublerulesep}{\arrayrulewidth}
\usepackage{booktabs}

\usepackage{hyperref}
\hypersetup{
    colorlinks=true,
    linkcolor=blue,
    filecolor=magenta,      
    urlcolor=blue,
    citecolor=blue,    
}

\newcommand{\note}[1]{
	\begin{center}
		\huge{ \textcolor{red}{#1} }
	\end{center}
}

\setcounter{topnumber}{2}
\setcounter{bottomnumber}{2}
\setcounter{totalnumber}{4}
\renewcommand{\topfraction}{0.85}
\renewcommand{\bottomfraction}{0.85}
\renewcommand{\textfraction}{0.15}
\renewcommand{\floatpagefraction}{0.8}
\renewcommand{\textfraction}{0.1}
\setlength{\floatsep}{5pt plus 2pt minus 2pt}
\setlength{\textfloatsep}{5pt plus 2pt minus 2pt}
\setlength{\intextsep}{5pt plus 2pt minus 2pt}

\newcommand{\quotes}[1]{``#1''}
\usepackage{array}
\newcolumntype{C}[1]{>{\centering\let\newline\\\arraybackslash\hspace{0pt}}m{#1}}
\usepackage[american]{circuitikz}
\usetikzlibrary{calc}
\usepackage{fancyhdr}
\usepackage{units} 

\graphicspath{{../Ejercicio-1/}{../Ejercicio-2/}{../Ejercicio-3/}{../Ejercicio-4/}}

\pagestyle{fancy}
\fancyhf{}
\lhead{22.14 - Electrónica IV}
\rhead{Mechoulam, Lambertucci, Londero}
\rfoot{Página \thepage}


\begin{document}

\begin{flushleft}
\begin{thebibliography}{9}

\bibitem{ref:final}
\quotes{Limit switch - Wikipedia}, En.wikipedia.org, 2021. [Online]. Disponible: \href{https://en.wikipedia.org/wiki/Limit\_switch}{https://en.wikipedia.org/wiki/Limit\_switch}. [Accedido: 28 Agosto 2021].

\end{thebibliography}
\end{flushleft}

\end{document}


\end{document}
