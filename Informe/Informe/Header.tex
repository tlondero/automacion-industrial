\documentclass[a4paper]{article}
\usepackage[utf8]{inputenc}
\usepackage[spanish, es-tabla, es-noshorthands]{babel}
\usepackage[table,xcdraw]{xcolor}
\usepackage[a4paper, footnotesep=1.25cm, headheight=1.25cm, top=2.54cm, left=2.54cm, bottom=2.54cm, right=2.54cm]{geometry}
%\geometry{showframe}

%\usepackage{wrapfig}			%Wrap figure in text
\usepackage[export]{adjustbox}	%Move images
\usepackage{changepage}			%Move tables

\usepackage{tikz}
\usepackage{amsmath}
\usepackage{amsfonts}
\usepackage{amssymb}
\usepackage{float}
\usepackage{graphicx}
\usepackage{caption}
\usepackage{subcaption}
\usepackage{multicol}
\usepackage{multirow}
\usepackage{wrapfig}
\setlength{\doublerulesep}{\arrayrulewidth}
\usepackage{booktabs}
\usepackage[numbib, nottoc, notlot, notlof]{tocbibind}

\usepackage{hyperref}
\hypersetup{
    colorlinks=true,
    linkcolor=blue,
    filecolor=magenta,      
    urlcolor=blue,
    citecolor=blue,    
}

%Change Font Size

% #1 = size, #2 = text
\newcommand{\setparagraphsize}[2]{{\fontsize{#1}{6}\selectfont#2 \par}}		%Cambia el size de todo el parrafo
\newcommand{\setlinesize}[2]{{\fontsize{#1}{6}\selectfont#2}}				%Cambia el font de una oración

\newcommand{\note}[1]{
	\begin{center}
		\huge{ \textcolor{red}{#1} }
	\end{center}
}

%FONTS (IMPORTANTE): Compilar en XeLaTex o LuaLaTeX
\usepackage{anyfontsize}	%Font size
\usepackage{fontspec}		%Font type

\setcounter{topnumber}{2}
\setcounter{bottomnumber}{2}
\setcounter{totalnumber}{4}
\renewcommand{\topfraction}{0.85}
\renewcommand{\bottomfraction}{0.85}
\renewcommand{\textfraction}{0.15}
\renewcommand{\floatpagefraction}{0.8}
\renewcommand{\textfraction}{0.1}
\setlength{\floatsep}{5pt plus 2pt minus 2pt}
\setlength{\textfloatsep}{5pt plus 2pt minus 2pt}
\setlength{\intextsep}{5pt plus 2pt minus 2pt}

\newcommand{\quotes}[1]{``#1''}
\usepackage{array}
\newcolumntype{C}[1]{>{\centering\let\newline\\\arraybackslash\hspace{0pt}}m{#1}}
\usepackage[american]{circuitikz}
\usetikzlibrary{calc}
\usepackage{fancyhdr}
\usepackage{units} 

\graphicspath{{../Control de posición no lineal/}{../Control de fuerza no lineal/}{../Control híbrido no lineal/}{../Referencias/}{../Deducción de modelo/}}

\pagestyle{fancy}
\fancyhf{}
\lhead{22.99 - Automación Industrial}
\rhead{Lambertucci, Londero B., Maselli, Mechoulam}
\rfoot{Página \thepage}
