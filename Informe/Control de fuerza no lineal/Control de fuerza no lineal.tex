%\documentclass[a4paper]{article}
\usepackage[utf8]{inputenc}
\usepackage[spanish, es-tabla, es-noshorthands]{babel}
\usepackage[table,xcdraw]{xcolor}
\usepackage[a4paper, footnotesep = 1cm, width=22cm, top=2.5cm, height=25cm, textwidth=20cm, textheight=25cm]{geometry}
%\geometry{showframe}

\usepackage{tikz}
\usepackage{amsmath}
\usepackage{amsfonts}
\usepackage{amssymb}
\usepackage{float}
\usepackage{graphicx}
\usepackage{caption}
\usepackage{subcaption}
\usepackage{multicol}
\usepackage{multirow}
\usepackage{wrapfig}
\setlength{\doublerulesep}{\arrayrulewidth}
\usepackage{booktabs}

\usepackage{hyperref}
\hypersetup{
    colorlinks=true,
    linkcolor=blue,
    filecolor=magenta,      
    urlcolor=blue,
    citecolor=blue,    
}

\newcommand{\note}[1]{
	\begin{center}
		\huge{ \textcolor{red}{#1} }
	\end{center}
}

\setcounter{topnumber}{2}
\setcounter{bottomnumber}{2}
\setcounter{totalnumber}{4}
\renewcommand{\topfraction}{0.85}
\renewcommand{\bottomfraction}{0.85}
\renewcommand{\textfraction}{0.15}
\renewcommand{\floatpagefraction}{0.8}
\renewcommand{\textfraction}{0.1}
\setlength{\floatsep}{5pt plus 2pt minus 2pt}
\setlength{\textfloatsep}{5pt plus 2pt minus 2pt}
\setlength{\intextsep}{5pt plus 2pt minus 2pt}

\newcommand{\quotes}[1]{``#1''}
\usepackage{array}
\newcolumntype{C}[1]{>{\centering\let\newline\\\arraybackslash\hspace{0pt}}m{#1}}
\usepackage[american]{circuitikz}
\usetikzlibrary{calc}
\usepackage{fancyhdr}
\usepackage{units} 

\graphicspath{{../Ejercicio-1/}{../Ejercicio-2/}{../Ejercicio-3/}{../Ejercicio-4/}}

\pagestyle{fancy}
\fancyhf{}
\lhead{22.14 - Electrónica IV}
\rhead{Mechoulam, Lambertucci, Londero}
\rfoot{Página \thepage}


%\begin{document}

\subsection{Caracterizaci\'on del problema}
Para este caso se pidi\'o un control de fuerzas no lineal. Para ello un aspecto fundamental es modelar las fuerzas del sistema que interaccionan con el actuador.
En este caso es la pared que trabaja como obst\'aculo. La fuerza de reacci\'on entre el EE y la pared ser\'a aquella definida por:
\begin{equation}
f_r = k_e \cdot d \ \hat{n}
\end{equation}
La cual es una fuerza proporcional a la distancia y normal a la superficie de la pared.
Esta fuerza  ser\'a nula mientras el EE no se encuentre en contacto con la pared y no nula en caso de estarlo.

\subsection{Esquema de control propuesto}
El esquema de control porpuesto ser\'a nuevamente una linealizaci\'on por realimentaci\'on de la siguiente manera.
\begin{figure}[H]
	\centering
	\includegraphics[width=0.8\linewidth]{ImagenesControl de fuerza no lineal/controlf}
	\caption{Topolog\'ia del control de fuerza no lineal.}	
	\label{fig:control_f_modelo}
\end{figure}


HABLAR DE VALROES DE GANANCIAS
\subsection{Resultados}
Se realiz\'o el simulink del sistema. Obteniendo los siguientes gr\'aficos.
Algo carácteristico que tiene este control de fuerzas es que debido a que no hay información provista como referencia de posición, el manipulador se moverá hasta estar cerca de la pared, y leugo realizará una breve oscilación sobre la pared hasta llegar a un estado permanente en el cual el manipulador estará apicando la fuerza comandada por la referencia.
\begin{figure}[H]
	\centering
	\includegraphics[width=0.8\linewidth]{ImagenesControl de fuerza no lineal/2_3_a}
	\caption{\'Angulos en funci\'on del tiempo en espacio joint.}	
	\label{fig:athetas}
\end{figure}
Se observa claramente en la siguiente imagen, como oscila levemente el EE al llegar a la pared.
\begin{figure}[H]
	\centering
	\includegraphics[width=0.8\linewidth]{ImagenesControl de fuerza no lineal/2_3_b}
	\caption{Posici\'on  del EE.}	
	\label{fig:apos}
\end{figure}
\begin{figure}[H]
	\centering
	\includegraphics[width=0.5\linewidth]{ImagenesControl de fuerza no lineal/2_3_c}
	\caption{Gr\'afico XY.}	
	\label{fig:axy}
\end{figure}
Cabe mencionar en este gráfico que en todos los puntos en los cuales la función es cero, se debe a que el manipulador no esta en contacto con la pared por lo que la misma no aplica una fuerza.
Además se ve como una vez alcanzada la pared y dada una referencia, el EE osicla levemente en torno a la referencia hasta establecerse.
\begin{figure}[H]
	\centering
	\includegraphics[width=0.8\linewidth]{ImagenesControl de fuerza no lineal/2_3_e}
	\caption{Gr\'afico XY.}	
	\label{fig:af}
\end{figure}
Ademas se le incluy\'o un disturbio a la planta tanto en posici\'on como en velocidad.
Se puede ver claramente en este gráfico el momento en el que se le aplica el disturbio a la planta, y como las variables de estado se estabilizan nuevamente.
\begin{figure}[H]
	\centering
	\includegraphics[width=0.8\linewidth]{ImagenesControl de fuerza no lineal/2_3_f_a}
	\caption{\'Angulos en funci\'on del tiempo en espacio joint.}	
	\label{fig:athetasd}
\end{figure}
Aqui se ve el movimiento del EE.
\begin{figure}[H]
	\centering
	\includegraphics[width=0.8\linewidth]{ImagenesControl de fuerza no lineal/2_3_f_b}
	\caption{Posici\'on del EE.}	
	\label{fig:aposd}
\end{figure}
\begin{figure}[H]
	\centering
	\includegraphics[width=0.5\linewidth]{ImagenesControl de fuerza no lineal/2_3_f_c}
	\caption{Gr\'afico XY.}	
	\label{fig:bxyd}
\end{figure}
Y finalmente se observa un gráfico similar al de fuerzas anterior, con la diferencia de que se hace nulo el valor de la fuerza aplicada en el segundo 15 al EE despegarse completamente de la pared. luego vuelve a la misma y se establece en el régimen permanente nuevamente.
\begin{figure}[H]
	\centering
	\includegraphics[width=0.8\linewidth]{ImagenesControl de fuerza no lineal/2_3_f_e}
	\caption{Fuerza deseada y real.}	
	\label{fig:bfd}
\end{figure}
%\end{document}
